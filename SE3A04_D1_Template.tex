\documentclass[]{article}

% Imported Packages
%------------------------------------------------------------------------------
\usepackage{amssymb}
\usepackage{amstext}
\usepackage{amsthm}
\usepackage{amsmath}
\usepackage{enumerate}
\usepackage{fancyhdr}
\usepackage[margin=1in]{geometry}
\usepackage{graphicx}
%\usepackage{extarrows}
%\usepackage{setspace}
%\usepackage{xcolor}
\usepackage{color}
%------------------------------------------------------------------------------

% Header and Footer
%------------------------------------------------------------------------------
\pagestyle{plain}  
\renewcommand\headrulewidth{0.4pt}                                      
\renewcommand\footrulewidth{0.4pt}                                    
%------------------------------------------------------------------------------

% Title Details
%------------------------------------------------------------------------------
\title{Deliverable \#1 : Software Requirement Specification (SRS)}
\author{SE 3A04: Software Design III -- Large System Design}
\date{}
                            
%------------------------------------------------------------------------------

% Document
%------------------------------------------------------------------------------
\begin{document}

\maketitle	
\noindent{\bf Tutorial Number:} T01\\
{\bf Group Number:} G01 \\
{\bf Group Members:} 
\begin{itemize}
	\item Benjamin Bloomfield
	\item Vikram Chandar
    \item Trisha Panchal
    \item Yug Vashisth
    \item Hamza Nadeem
\end{itemize}

\section*{IMPORTANT NOTES}
\begin{itemize}
	\item Be sure to include all sections of the template in your document regardless whether you have something to write for each or not
	\begin{itemize}
		\item If you do not have anything to write in a section, indicate this by the \emph{N/A}, \emph{void}, \emph{none}, etc.
	\end{itemize}
	\item Uniquely number each of your requirements for easy identification and cross-referencing
	\item Highlight terms that are defined in Section~1.3 (\textbf{Definitions, Acronyms, and Abbreviations}) with \textbf{bold}, \emph{italic} or \underline{underline}
	\item For Deliverable 1, please highlight, in some fashion, all (you may have more than one) creative and innovative features. Your creative and innovative features will generally be described in Section~2.2 (\textbf{Product Functions}), but it will depend on the type of creative or innovative features you are including.
\end{itemize}

\newpage
\section{Introduction}
\label{sec:introduction}
This Software Requirements Specification (SRS) defines the requirements for the Smart City Environmental Monitoring \& Alert System (SCEMAS), a cloud-based platform for smart-city environmental monitoring and alerting. SCEMAS collects real-time environmental telemetry from a distributed sensor network, ingests it through an MQTT-based interface, validates incoming messages, and stores the data for both real-time and historical analysis. The platform aggregates readings, evaluates configurable threshold rules, and generates alerts to support timely city response. SCEMAS provides a secure operator dashboard for visualization and exposes a read-only REST API that returns aggregated, non-sensitive environmental data for public and third-party consumption. This document establishes the system scope, key stakeholders and users, core capabilities, and the constraints and quality attributes that guide implementation and verification.


\begin{itemize}
	\item Provide an overview of the document/SRS.
\end{itemize}


\subsection{Purpose}
\label{sub:purpose}
% Begin SubSection
This Document is used to establish the requirements, performance, and intended behaviour/uses of the Smart City Environmental Monitoring and Alert System (SCEMAS).
The intended audience of the SRS is the stakeholders of SCEMAS, including developers and testers.
% End SubSection

\subsection{Scope}
\label{sub:scope}
% Begin SubSection
\begin{itemize}
	\item Identify the software product(s) to be produced, and name each (e.g., Host DBMS, Report Generator, etc.)
	\item Explain what the software product(s) will do (and, if necessary, also state what they will not do).
	\item Describe the application of the software being specified, including relevant benefits, objectives, and goals.
%	\item Be consistent with similar statements in higher-level specifications (e.g., the system requirements specification), if they exist
\end{itemize}
% End SubSection

\subsection{Definitions, Acronyms, and Abbreviations}
\label{sub:definitions_acronyms_and_abbreviations}
% Begin SubSection
\begin{itemize}
	\item Provide the definitions of all terms, acronyms, and abbreviations required to properly interpret the SRS.
	\item This should be in alphabetical order.
\end{itemize}
% End SubSection

\subsection{References}
\label{sub:references}
% Begin SubSection
\begin{itemize}
	\item Provide a complete list of all documents referenced elsewhere in the SRS.
	\item Identify each document by title, report number (if applicable), date, and publishing organization.
	\item Specify the sources from which the references can be obtained.
	\item Order this list in some sensible manner (alphabetical by author, or something else that makes more sense).
\end{itemize}
% End SubSection

\subsection{Overview}
\label{sub:overview}
The remainder of this Software Requirements Specification (SRS) is organized as follows. Section 2 provides an overall description of SCEMAS, including its context, major functions, intended users, constraints, and assumptions. Section 3 presents the system’s use case diagram. Section 4 outlines the functional requirements using the main business events and viewpoint-based scenarios, ending with a consolidated global scenario. Section 5 describes the non-functional requirements, including usability, performance, operational/environmental constraints, maintainability/support, security, cultural/political, and legal considerations. Finally, the Appendix includes the division of labor and any supporting material required by the course template.
\end{itemize}
% End SubSection

% End Section

\section{Overall Product Description}
\label{sec:overall_description}
% Begin Section

\begin{itemize}
	\item This section should describe the general factors that affect the product and its requirements. 
	\item It does not state specific requirements.
	\item It provides a \emph{background} for those requirements and makes them easier to understand.
\end{itemize}


\subsection{Product Perspective}
\label{sub:product_perspective}
% Begin SubSection

SCEMAS is a cloud-native IoT software platform built for smart-city environmental monitoring and alerting. It is similar in purpose to Government of Canada environmental dashboards such as the Canadian Environmental Sustainability Indicators (CESI) (\texttt{https://indicators-map.canada.ca/}) interactive indicators and maps, as well as national monitoring dashboards that report trends and exceedances (for example, water monitoring and wastewater surveillance). However, SCEMAS is designed for city operations where near real-time telemetry ingestion, automated threshold-based alerting, and zone-level aggregation are needed to support fast response and day-to-day decision-making. In many cities, environmental data is tracked in a scattered or manual way, which can slow down responses to issues like poor air quality, heatwaves, or noise disturbances. SCEMAS addresses this by bringing everything into one place: it collects telemetry in real time, automatically detects when conditions cross important thresholds, and shares clear, timely information with both city operators and the public.

At a high level, SCEMAS sits between a network of environmental sensing devices and the people or systems that need environmental insights. During development and testing, the sensors act like real devices by publishing telemetry messages into the platform. Incoming telemetry is accepted through an MQTT-based ingestion entry point, then passed through a validation and ingestion component that checks message structure and plausible value ranges before storing the data in persistent time-series storage.

Once data is stored, SCEMAS performs real-time aggregation to produce useful summaries for monitoring and reporting. A configurable, rule-based alerting engine continuously evaluates these readings and aggregates to detect threshold violations and track alert status over time. When an alert is triggered, SCEMAS can notify external subscriber systems, while also presenting the situation to operators through a secure web dashboard that includes maps, charts, current values, active alerts, system health, and historical trends.

SCEMAS also has a read-only REST API which can be provided to approved third-party organizations upon request. The API returns aggregated, non-sensitive environmental data, and can be paired with a simple demonstration client showing how to consume the endpoints. Potential consumers include mapping applications (e.g., Google Maps overlays) and external organizations that want to display SCEMAS data for their own users or services. To ensure secure administration and accountability, the platform includes role-based access control (RBAC) and audit logging across operator-facing functions and administrative actions.

\begin{figure}[1]
    \centering
    \includegraphics[width=0.95\linewidth]{F1Block_D.png}
    \caption{SCEMAS high-level block diagram and external interfaces.}
    \label{fig:scemas_block_diagram}
\end{figure}

% End SubSection

\subsection{Product Functions}
\label{sub:product_functions}
% Begin SubSection
\begin{itemize}
    \item Accurate data Processing: The data processed by the system must be accurate with an error rate of less than 1 percent.
    \item Data Processing: The Data must validate each incoming message.
    \item Efficient Storage: The system shall store data in a way that will support efficient querying.
    \item Alert Engine: The system must have a configurable rule-based alert engine with definable alert rules based on specific thresholds.
    \item Alert History: The System shall keep a comprehensive history of all alerts that have occurred.
    \item Dashboard: The System shall have a dashboard that shows visual representations of metrics for city operators
    \item Public Dashboard: The System shall provide access to non sensitive data for public use.
    \item Role-based Permissions: Each user should have an assigned role that dictates what they may do with the system.
    \item The system shall maintain an audit log of all significant events.
    \item Authentication: The system shall authenticate all interacting entities before accessing data. 
    \item Privacy: The System shall not collect, store, or process any personal identifiable information (PII) 
    \item System Efficiency: The system shall provide efficient response times to api requests. 
    
\end{itemize}

% End SubSection

\subsection{User Characteristics}
\label{sub:user_characteristics}

\begin{itemize}
	\item Describe those general characteristics of the intended users of the product including educational level, experience, and technical expertise 
	\item Since there will be many users, you may wish to divide into different user types or personas
	\item Do not state specific requirements, but rather provide the reasons why certain specific requirements are later specified
\end{itemize}

\subsubsection{User I: City Operators}
City operators analyze the web dashboard to monitor key environmental conditions, and respond accordingly to alerts generated by the system.
\begin{itemize}
    \item \textbf{Education Level}: City operators are expected to have some level of post-secondary education, in environmental science, engineering, urban planning, or another related field of study.
    \item \textbf{Experience}: Prior experience  working in a similar role, such as environmental monitoring or public safety operations. Such users should be familiar with interpreting geographical maps, environmental metrics, charts, and trends to identify irregular environmental conditions.
    \item \textbf{Technical Expertise}: City operators are expected to have intermediate computer proficiency, such as the ability to navigate web-based applications, interact with dashboards, and operate data visualization tools.
\end{itemize}

\subsubsection{User II: Public Users}
The general public and citizens can access non-sensitive environmental information through public interfaces that use the read-only REST API, for awareness and safety purposes.
\begin{itemize}
    \item \textbf{Education Level}: Public users are assumed to have basic literacy skills. Basic foundations in listening, reading, writing, and speaking are all the user needs to use the platform without any challenges.
    \item \textbf{Experience}: The user is  expected to have little to no prior experience using monitoring systems. The system is designed to be user-friendly to ensure that interactions are straightforward for any user.
    \item \textbf{Technical Expertise}: Basic technical skills, such as understanding how to use a web browser, navigate a webpage, and view digital displays is all that is necessary.
\end{itemize}

\subsubsection{User III: Administrators}
System administrators are responsible for building and maintaining the system. Their role focuses specifically on ensuring the system operates securely, behaves correctly, and is maintainable over time.
\begin{itemize}
    \item \textbf{Education Level}: System administrators should have at least a post-secondary education in software engineering, computer science, or another similar technical field.
    \item \textbf{Experience}: They are expected to have prior experience maintaining or managing software systems in a professional or academic setting. This experience includes responsibilities related to access control, system configuration and operational oversight. Such users also should understand how to define system policies and supervise automated processes like logging and alert handling. 
    \item \textbf{Technical Expertise}: Advanced technical skills are required for system administrators. They should be familiar with configurable rule-based systems, authentication methods, role-based access control (RBAC) models, and audit logging.
\end{itemize}

\subsubsection{User IV: Third-Party Developer}
Third-party developers request and interpret environmental data from the public API to help build external applications. 
\begin{itemize}
    \item \textbf{Education Level}: Developers should have a post-secondary education in software engineering, computer science, or equivalent professional experience.
    \item \textbf{Experience}: The third-party developers are expected to have experience developing software that uses REST APIs and works with JSON data
    \item \textbf{Technical Expertise}: Proficient technical skills in web application development and API integration, as well as the ability to comprehend API documentation.
\end{itemize}
% End SubSection

\subsection{Constraints}
\label{sub:constraints}
% Begin SubSection
\begin{itemize}
	\item Provide a general description of any constraints that will limit the developer's options
\end{itemize}

\begin{enumerate}
    \item \textbf{Budget}: The project is constrained by a budget of \$0. This impacts what features the developer can implement, as all technologies, frameworks, cloud services, and other development tools must be free for the team to use.
    \item \textbf{Project Time}: The development of the system is constrained to the Winter 2026 academic calendar. This limits the overall scope of the system, and the amount of features that can be implemented at launch. It shapes the overall development timeline, and the ability to test and optimize. 
    \item \textbf{Communication Protocol}: All sensor telemetry must use the MQTT protocol. This restricts the system’s communication model and requires that incoming messages align with the predefined data formats that MQTT messaging supports.
    \item \textbf{Personal Identifiable Information}: The system is constrained to operate under privacy-by-design principles, and must not collect, process, or store any personally identifiable information. Environmental data must be aggregated and associated with general city zones rather than precise locations.
    \item \textbf{Public API Scope}: The public and third-party facing API is constrained to be read-only, meaning that they can not submit data or modify the state. It must only provides access to aggregated, non-sensitive environmental data.
    \item \textbf{Simulated Hardware}: Physical IoT sensor hardware is not available for this project and must be be generated using software simulations. This influences design decisions as the functionality cannot be verified using real-world sensors.
    \item \textbf{Deployment Environment}: The system is constrained to a cloud architecture using available platforms and tools suitable for educational use. Therefore, developers cannot deploy the system on local machines or private servers.
\end{enumerate}
% End SubSection

\subsection{Assumptions and Dependencies}
\label{sub:assumptions_and_dependencies}
% Begin SubSection
\begin{itemize}
	\item List any assumptions you made in interpreting what the software being developed is aiming to achieve
	\item List any other assumptions you made that, if it fails to hold, could require you to change the requirements
	%\item List each of the factors that affect the requirements stated in the SRS
	%\item These factors are not design constraints on the software but are, rather, any changes to them that can affect the requirements in the SRS
	\begin{enumerate}
        \item IoT Sensors
        \begin{enumerate}
            \item Environmental Data
            \begin{enumerate}
                \item IoT sensors provide accurate environmental information to the web application.
                \item IoT sensors provide data in real-time.
            \end{enumerate}
            \item Deployment
            \begin{enumerate}
                \item All IoT sensors have been already installed and require no setup.
            \end{enumerate}
            \item Messaging
            \begin{enumerate}
                \item All telemetry messages follow the predefined MQTT message schema agreed upon by the system.
            \end{enumerate}
        \end{enumerate}
    
        \item Assumptions
        \begin{enumerate}
            \item It is assumed that no PII will be collected or transmitted by IoT sensors.
            \item It is assumed that the selected cloud provider provides sufficient uptime, scalability, and availability for development.
            \item The system assumes that it has a database capable of efficiently storing and querying time-series data.
        \end{enumerate}
    
        \item Scope Limitations
        \begin{enumerate}
            \item Dashboards and IoT Sensor Integration is limited to one city located in Canada.
            \item The SCHEMAS dashboard shall display accurately on all web browsers.
        \end{enumerate}
    \end{enumerate}
    
\end{itemize}
% End SubSection

\subsection{Apportioning of Requirements}
\label{sub:apportioning_of_requirements}
% Begin SubSection
\begin{itemize}
	\item Identify requirements that may be delayed until future versions of the system
\end{itemize}
% End SubSection

% End Section
\section{Use Case Diagram}
\label{sec:use_case_diagram}
% Begin Section
\begin{itemize}
	\item Provide the use case diagram for the system being developed.
	\item You do not need to provide the textual description of any of the use cases here (these will be specified under "Highlights of Functional Requirements").
%	\item Provide \emph{one} use case diagram for the most important Business Event.
%	\item The text of all use cases will be specified under "Highlights of Functional Requirements"
\end{itemize}
%In this section, select the most important Business Event that your system responds to and give its use case diagram.  Only one use case diagram is needed.  Give a brief textual description of the use case without repeating what is in the scenarios of the corresponding Business Event.

%
%
%
%This section should provide a use case diagram for your application. 
%\begin{enumerate}[a)]
%	\item Each use case appearing in the diagram should be accompanied by a text description. 
%\end{enumerate}
%% End Section

\section{Highlights of Functional Requirements}
\label{sec:functional_requirements}
% Begin Section
\begin{itemize}
	\item Specify all use cases (or other scenarios triggered by other events), organized by Business Event. 
	\item For each Business Event, show the scenario from every Viewpoint. You should have the same set of Viewpoints across all Business Events. If a Viewpoint doesn't participate, write N/A so we know you considered it still. You can choose how to present this - keep in mind it should be easy to follow. 
	\item At the end, combine them all into a Global Scenario.
	%\item Specify the "use cases" (or other triggering events) organized by Business Event. (The Global Scenario is what you might think of as a use case). Be sure to consider Business Events that aren't just triggered by users with goals (e.g. something happens in the environment that your system needs to respond to)
	\item Your focus should be on what the system needs to do, not how to do it. Specify it in enough detail that it clearly specifies what needs to be accomplished, but not so detailed that you start programming or making design decisions.
	\item Keep the length of each use case (Global Scenario) manageable. If it's getting too long, split into sub-cases.
	\item You are \emph{not} specifying a complete and consistent set of functional requirements here. (i.e. you are providing them in the form of use cases/global scenarios, not a refined list). For the purpose of this project, you do not need to reduce them to a list; the global scenarios format is all you need.
	\item Red text below is just to highlight where you need to insert a scenario - don't actually write it all in red.
\end{itemize}

\noindent {\bf Main Business Events:} The business events that relate to the system include:
\begin{itemize}
    \item BE1. Define new alert rule.
    \item BE2. Update existing alert rule.
    \item BE3. Remove or disable an alert rule.
    \item BE4. Confirm alert.
    \item BE5. Mark the alert as resolved
    \item BE6. Request Access to Public API
    \item BE7. Modify Account
\end{itemize}


\noindent {\bf Viewpoints:} The viewpoints the system considers include:
\begin{itemize}
    \item VP1. City Operators
    \item VP2. Public Users
    \item VP3. Administrators
    \item VP4. Third Party Developers 
    \item VP5. Environmental Monitoring Department
    \item VP6. IT Support
\end{itemize}

\noindent {\bf Business Events:} 

\begin{enumerate}[{\bf BE1.}]
	\item Define a New Alert Rule. \#1 \\         
    \textbf{Pre-condition}: The administrator has an authenticated account and there is no alert rule that already exists for the same threshold.
    
		\begin{enumerate}[{\bf VP1.}]
			\item City Operators \#1 
                \begin{quote}
                    N/A
                \end{quote}
                
			\item Public Users \#2 
				\begin{quote}
				    N/A
				\end{quote}
                
			\item Administrators \#3 \\
				\textbf{Main Success Scenario} 
                    \begin{quote}
                    \begin{enumerate}[1.]
                        \item Administrator opens the dashboard.
                        \item System prompts the administrator to log in.
                        \item Administrator enters login information.
                        \item System authenticates the administrator.
                        \item System displays the dashboard.
                        \item Administrator selects \textit{Manage Alerts} tab
                        \item System displays the existing alert rules, and the option to create a new rule.
                        \item Administrator selects \textit{Create New Alert Rule}.
                        \item System displays the alert rule setup form.
                        \item Administrator enters the new alert rule information.
                        \item Administrator submits the alert rule to be added to the system.
                        \item System validates the form.
                        \item System stores the new alert rule.
                        \item System reports to administrator that the alert has been successfully created.
                    \end{enumerate}
                    \end{quote}
                \textbf{Secondary Scenario} 
                \begin{quote}
                    \begin{description}
                        \item[\textmd{4i.}] System fails to authenticate the administrator.
                        \begin{description}
                            \item[\textmd{4i.1}] System rejects the authentication attempt.
                            \item[\textmd{4i.2}] System displays an authentication failure message.
                            \item[\textmd{4i.3}] System prompts the administrator to re-enter credentials.
                            \item[\textmd{4i.4}] Administrator selects try again.
                            \item[\textmd{4i.5}] System tracks the number of consecutive failed sign-in attempts.
                            \item[\textmd{4i.6}] Return to BE1-3.
                        \end{description}
                        \item[\textmd{4ii.}] Maximum amount of sign-in attempts have been reached. 
                        \begin{description}
                            \item[\textmd{4ii.1}] System temporarily locks the account.
                            \item[\textmd{4ii.2}] System displays a message indicating that access is restricted.
                            \item[\textmd{4ii.3}] System provides instructions for contacting IT Support to resolve the issue.
                        \end{description}
                        \item[\textmd{12i.}] System detects invalid or incomplete data.
                        \begin{description}
                            \item[\textmd{12i.1}] System prevents the alert rule from being submitted.
                            \item[\textmd{12i.2}] System displays an error message describing the issue.
                            \item[\textmd{12i.3}] System highlights the section(s) that need to be changed.
                            \item[\textmd{12i.4}] Return to BE1-10.
                        \end{description}
                        \item[\textmd{13i.}] System fails to store the added alert rule.
                        \begin{description}
                            \item[\textmd{13i.1}] System displays an error message that the alert rule could not be created.
                            \item[\textmd{13i.2}] System provides instructions for contacting IT Support.
                        \end{description}
                    \end{description}
                \end{quote}
                
			\item Third-Party Developers \#4 
				\begin{quote}
				    N/A
				\end{quote}
                
			\item Environmental Monitoring Department \#5 
				\begin{quote}
				    N/A
				\end{quote}

			\item IT Support \#6 
                \item[\textmd{4ii}] System should allow IT Support to restore administrator access after verifying their identity.
                \item[\textmd{13i}] System should provide IT Support with error logs to resolve the alert rule storage failures.
		\end{enumerate}
        
        {\bf Global Scenario:}
        	\begin{quote}
        	    \textbf{Pre-condition}: Define a New Alert Rule.. \\
                \textbf{Main Success Scenario} 
                    \begin{quote}
                    \begin{enumerate}[1.]
                        \item Administrator opens the dashboard.
                        \item System prompts the administrator to log in.
                        \item Administrator enters login information.
                        \item System authenticates the administrator.
                        \item System displays the dashboard.
                        \item Administrator selects \textit{Manage Alerts} tab
                        \item System displays the existing alert rules, and the option to create a new rule.
                        \item Administrator selects \textit{Create New Alert Rule}.
                        \item System displays the alert rule setup form.
                        \item Administrator enters the new alert rule information.
                        \item Administrator submits the alert rule to be added to the system.
                        \item System validates the form.
                        \item System stores the new alert rule.
                        \item System reports to administrator that the alert has been successfully created.
                    \end{enumerate}
                    \end{quote}
                \textbf{Secondary Scenario} 
                \begin{quote}
                    \begin{description}
                        \item[\textmd{4i.}] System fails to authenticate the administrator.
                        \begin{description}
                            \item[\textmd{4i.1}] System rejects the authentication attempt.
                            \item[\textmd{4i.2}] System displays an authentication failure message.
                            \item[\textmd{4i.3}] System prompts the administrator to re-enter credentials.
                            \item[\textmd{4i.4}] Administrator selects try again.
                            \item[\textmd{4i.5}] System tracks the number of consecutive failed sign-in attempts.
                            \item[\textmd{4i.6}] Return to BE1-3.
                        \end{description}
                        \item[\textmd{4ii.}] Maximum amount of sign-in attempts have been reached. 
                        \begin{description}
                            \item[\textmd{4ii.1}] System temporarily locks the account.
                            \item[\textmd{4ii.2}] System displays a message indicating that access is restricted.
                            \item[\textmd{4ii.3}] System provides instructions for contacting IT Support.
                        \end{description}
                        \item[\textmd{12i.}] System detects invalid or incomplete data.
                        \begin{description}
                            \item[\textmd{12i.1}] System prevents the alert rule from being submitted.
                            \item[\textmd{12i.2}] System displays an error message describing the issue.
                            \item[\textmd{12i.3}] System highlights the section(s) that need to be changed.
                            \item[\textmd{12i.4}] Return to BE1-10.
                        \end{description}
                        \item[\textmd{13i.}] System fails to store the added alert rule.
                        \begin{description}
                            \item[\textmd{13i.1}] System displays an error message that the alert rule could not be created.
                            \item[\textmd{13i.2}] System provides instructions for contacting IT Support.
                        \end{description}
                    \end{description}
                \end{quote}
        	\end{quote}
\end{enumerate}

\begin{enumerate}[{\bf BE2.}]
	\item Update an Existing Alert Rule. \#2 \\         
    \textbf{Pre-condition}: There must exist at least one alert rule in the system. The administrator is authenticated and logged into the system.
    
		\begin{enumerate}[{\bf VP1.}]
			\item City Operators \#1 
                \begin{quote}
                    N/A
                \end{quote}
                
			\item Public Users \#2 
				\begin{quote}
				    N/A
				\end{quote}
                
			\item Administrators \#3 \\
				\textbf{Main Success Scenario} 
                    \begin{quote}
                    \begin{enumerate}[1.]
                        \item System displays the dashboard.
                        \item Administrator selects \textit{Manage Alerts} tab
                        \item System displays the existing alert rules.
                        \item Administrator selects an alert rule to update.
                        \item Administrator modifies the alert rule parameters.
                        \item Administrator submits the updated alert rule.
                        \item System validates the updated alert rule.
                        \item System updates the alert rule in the systems database.
                        \item System reports to the administrator that the alert rule has been successfully updated.
                    \end{enumerate}
                    \end{quote}
                \textbf{Secondary Scenario} 
                \begin{quote}
                    \begin{description}
                        \item[\textmd{7i.}] Updated alert information is invalid or incomplete.
                        \begin{description}
                            \item[\textmd{7i.1}] System prevents the updated alert rule from being submitted.
                            \item[\textmd{7i.2}] System displays an error message describing the issue.
                            \item[\textmd{7i.3}] System highlights the section(s) that need to be changed.
                            \item[\textmd{7i.4}] Return to BE2-5.
                        \end{description}
                        \item[\textmd{8i.}] System fails to update the alert rule.
                        \begin{description}
                            \item[\textmd{8i.1}] System displays an error message indicating that the alert rule update failed.
                            \item[\textmd{8i.2}] System provides instructions for contacting IT Support.
                        \end{description}
                    \end{description}
                \end{quote}
                
			\item Third-Party Developers \#4
				\begin{quote}
				    N/A
				\end{quote}
                
			\item Environmental Monitoring Department \#5
				\begin{quote}
				    N/A
				\end{quote}

			\item IT Support \#6
            	\begin{quote}
				    8i. System provides IT Support with error logs to resolve the alert rule update failures.
				\end{quote}

		\end{enumerate}
        
        {\bf Global Scenario:}
        	\begin{quote}
        	    \textbf{Pre-condition}: There must exist at least one alert rule in the system. The administrator is authenticated and logged into the system. \\
                \textbf{Main Success Scenario} 
                    \begin{quote}
                    \begin{enumerate}[1.]
                        \item System displays the dashboard.
                        \item Administrator selects \textit{Manage Alerts} tab
                        \item System displays the existing alert rules.
                        \item Administrator selects an alert rule to update.
                        \item Administrator modifies the alert rule parameters.
                        \item Administrator submits the updated alert rule.
                        \item System validates the updated alert rule.
                        \item System updates the alert rule in the systems database.
                        \item System reports to the administrator that the alert rule has been successfully updated
                    \end{enumerate}
                    \end{quote}
                \textbf{Secondary Scenario} 
                \begin{quote}
                    \begin{description}
                        \item[\textmd{7i.}] Updated alert information is invalid or incomplete.
                        \begin{description}
                            \item[\textmd{7i.1}] System prevents the updated alert rule from being submitted.
                            \item[\textmd{7i.2}] System displays an error message describing the issue.
                            \item[\textmd{7i.3}] System highlights the section(s) that need to be changed.
                            \item[\textmd{7i.4}] Return to BE2-5.
                        \end{description}
                        \item[\textmd{8i.}] System fails to update the alert rule.
                        \begin{description}
                            \item[\textmd{8i.1}] System displays an error message that that the alert rule update failed.
                            \item[\textmd{8i.2}] System provides instructions for contacting IT Support.
                        \end{description}
                    \end{description}
                \end{quote}
        	\end{quote}
\end{enumerate}

\begin{enumerate}[{\bf BE3.}]
	\item Remove or Disable an Alert Rule \#3 \\         
    \textbf{Pre-condition}: There must exist at least one alert rule in the system. The administrator is authenticated and logged into the system.
    
		\begin{enumerate}[{\bf VP1.}]
			\item City Operators \#1 
                \begin{quote}
                    N/A
                \end{quote}
                
			\item Public Users \#2 
				\begin{quote}
				    N/A
				\end{quote}
                
			\item Administrators \#3 \\
				\textbf{Main Success Scenario} 
                    \begin{quote}
                    \begin{enumerate}[1.]
                        \item System displays the dashboard.
                        \item Administrator selects \textit{Manage Alerts} tab
                        \item System displays the existing alert rules.
                        \item Administrator selects an alert rule to remove or disable.
                        \item System prompts the administrator for confirmation.
                        \item Administrator confirms the action.
                        \item System removes or disables the alert rule.
                        \item System reports that the alert has been successfully removed or disabled.
                    \end{enumerate}
                    \end{quote}
                \textbf{Secondary Scenario} 
                \begin{quote}
                    \begin{description}
                        \item[\textmd{6i.}] Administrator cancels the confirmation.
                        \begin{description}
                            \item[\textmd{6i.1}] System does not remove or disable the alert rule.
                            \item[\textmd{6i.2}] System returns to the list of existing alert rules.
                        \end{description}
                        \item[\textmd{7i.}] System fails to remove or disable the alert rule.
                        \begin{description}
                            \item[\textmd{7i.1}] System displays an error message that the action failed.
                            \item[\textmd{7i.2}] System provides instructions for contacting IT Support.
                        \end{description}
                    \end{description}
                \end{quote}
                
			\item Third-Party Developers \#4
				\begin{quote}
				    N/A
				\end{quote}
                
			\item Environmental Monitoring Department \#5
				\begin{quote}
				    N/A
				\end{quote}

			\item IT Support \#6
                \item[\textmd{7i}] System provides IT Support with system logs to investigate and resolve alert rule removal or disable failures.
		\end{enumerate}
        
        {\bf Global Scenario:}
        	\begin{quote}
        	    \textbf{Pre-condition}: There must exist at least one alert rule in the system. The administrator is authenticated and logged into the system. \\
                \textbf{Main Success Scenario} 
                    \begin{quote}
                    \begin{enumerate}[1.]
                        \item System displays the dashboard.
                        \item Administrator selects \textit{Manage Alerts} tab
                        \item System displays the existing alert rules.
                        \item Administrator selects an alert rule to remove or disable.
                        \item System prompts user for confirmation.
                        \item Administrator confirms the action.
                        \item System removes or disables the alert rule.
                        \item System reports that the alert has been successfully removed or disabled.
                    \end{enumerate}
                    \end{quote}
                \textbf{Secondary Scenario} 
                \begin{quote}
                    \begin{description}
                        \item[\textmd{6i.}] Administrator cancels the confirmation.
                        \begin{description}
                            \item[\textmd{6i.1}] System does not remove or disable the alert rule.
                            \item[\textmd{6i.2}] System returns to the list of existing alert rules.
                        \end{description}
                        \item[\textmd{7i.}] System fails to remove or disable the alert rule.
                        \begin{description}
                            \item[\textmd{7i.1}] System displays an error message that the action failed.
                            \item[\textmd{7i.2}] System provides instructions for contacting IT Support.
                        \end{description}
                    \end{description}
                \end{quote}
        	\end{quote}
\end{enumerate}

\begin{enumerate}[{\bf BE4.}]
	\item City Operator Acknowledges an Alert \#4 \\      
    \textbf{Pre-condition}: An alert has been generated by the system and is marked as \textit{Active}. The city operator is authenticated and logged into the system.
    
		\begin{enumerate}[{\bf VP1.}]
			\item City Operators \#1 
                \begin{quote}
                    N/A
                \end{quote}
                
			\item Public Users \#2 
				\begin{quote}
				    N/A
				\end{quote}
                
			\item Administrators \#3 \\
				\textbf{Main Success Scenario} 
                    \begin{quote}
                    \begin{enumerate}[1.]
                        \item System displays the dashboard showing the active alerts.
                        \item City operator selects an active alert from the dashboard.
                        \item System displays the alert details.
                        \item City operator selects the option to acknowledge the alert.
                        \item System updates the alert status to \textit{Acknowledged}.
                        \item System confirms that the alert has been acknowledged. 
                    \end{enumerate}
                    \end{quote}
                \textbf{Secondary Scenario} 
                \begin{quote}
                    \begin{description}
                        \item[\textmd{4i.}] Alert has already been acknowledged by another operator.
                        \begin{description}
                            \item[\textmd{4i.1}] System refreshes the alert status.
                            \item[\textmd{4i.2}] System informs the city operator that the alert is already acknowledged.
                        \end{description}
                        \item[\textmd{5i.}] System fails to update the alert status.
                        \begin{description}
                            \item[\textmd{5i.1}] System displays an error message to city operator.
                            \item[\textmd{5i.2}] Alert remains in \textit{Active} status.
                            \item[\textmd{5i.3}] System provides instructions for contacting IT Support.
                        \end{description}
                    \end{description}
                \end{quote}
                
			\item Third-Party Developers \#4
				\begin{quote}
				    N/A
				\end{quote}
                
			\item Environmental Monitoring Department \#5
				\begin{quote}
				    N/A
				\end{quote}

			\item IT Support \#6
                \item[\textmd{5i}] System provides IT Support with system logs to investigate and resolve alert acknowledgment issues.
		\end{enumerate}
        
        {\bf Global Scenario:}
        	\begin{quote}
        	    \textbf{Pre-condition}: An alert has been generated by the system and is marked as \textit{Active}. The city operator is authenticated and logged into the system. \\
                \textbf{Main Success Scenario} 
                    \begin{quote}
                    \begin{enumerate}[1.]
                        \item System displays the dashboard showing the active alerts.
                        \item City operator selects an active alert from the dashboard.
                        \item System displays the alert details.
                        \item City operator selects the option to acknowledge the alert.
                        \item System updates the alert status to \textit{Acknowledged}.
                        \item System confirms that the alert has been acknowledged. 
                    \end{enumerate}
                    \end{quote}
                \textbf{Secondary Scenario} 
                \begin{quote}
                    \begin{description}
                        \item[\textmd{4i.}] Alert got acknowledged by another operator.
                        \begin{description}
                            \item[\textmd{4i.1}] System refreshes the alert status.
                            \item[\textmd{4i.2}] System informs the city operator that the alert is already acknowledged.
                        \end{description}
                        \item[\textmd{5i.}] System fails to update the alert status.
                        \begin{description}
                            \item[\textmd{5i.1}] System displays an error message to city operator.
                            \item[\textmd{5i.2}] Alert remains in \textit{Active} status.
                            \item[\textmd{5i.3}] System provides instructions for contacting IT Support.
                        \end{description}
                    \end{description}
                \end{quote}
        	\end{quote}
\end{enumerate}
\begin{enumerate}[{\bf BE5.}]
	\item Manage Role Permissions \#5 \\         
    \textbf{Pre-condition}: The user is authenticated. Administrators are authorized to manage roles and permissions.
    
		\begin{enumerate}[{\bf VP1.}]
			\item City Operators \#1 \\
                \textbf{Pre-condition}: City Operator is authenticated and has an active account.\\
                \textbf{Main Success Scenario}
                \begin{quote}
                    \begin{enumerate}[1.]
                        \item City Operator opens their profile or account settings.
                        \item System displays the City Operator’s assigned roles and a summary of allowed actions.
                        \item City Operator confirms what actions they are permitted to perform.
                    \end{enumerate}
                \end{quote}
                \textbf{Secondary Scenario}
                \begin{quote}
                    \begin{description}
                        \item[\textmd{3i.}] Permission change needed.
                        \begin{description}
                            \item[\textmd{3i.1}] City Operator submits an access change request describing the needed permission and reason.
                            \item[\textmd{3i.2}] System records the request and notifies an Administrator for review.
                        \end{description}
                        \item[\textmd{2i.}] Role/permission information unavailable.
                        \begin{description}
                            \item[\textmd{2i.1}] System cannot load role/permission information.
                            \item[\textmd{2i.2}] System displays an error message and instructs the user to retry or contact IT Support.
                        \end{description}
                    \end{description}
                \end{quote}
                
			\item Public Users \#2 
				\begin{quote}
				    N/A
				\end{quote}
                
			\item Administrators \#3 \\
                \textbf{Pre-condition}: Administrator is authenticated and authorized to manage roles and permissions.\\
				\textbf{Main Success Scenario} 
                    \begin{quote}
                    \begin{enumerate}[1.]
                        \item Administrator opens the role and permission management page.
                        \item System displays existing roles, permissions, and user-role assignments.
                        \item Administrator selects a role to create, edit, or remove.
                        \item Administrator adds or removes permissions for the selected role.
                        \item Administrator assigns or removes the role to users as needed.
                        \item System validates the changes against rules (e.g., preventing removal of required administrative access).
                        \item System saves changes and updates role assignments.
                        \item System records the change in an audit/history log and notifies affected users if applicable.
                    \end{enumerate}
                    \end{quote}
                \textbf{Secondary Scenario} 
                \begin{quote}
                    \begin{description}
                        \item[\textmd{6i.}] Invalid/unsafe permission change.
                        \begin{description}
                            \item[\textmd{6i.1}] Administrator attempts to apply a change that violates system rules.
                            \item[\textmd{6i.2}] System rejects the change and shows what must be corrected.
                        \end{description}
                        \item[\textmd{7i.}] Conflicting updates.
                        \begin{description}
                            \item[\textmd{7i.1}] Another Administrator modifies the same role at the same time.
                            \item[\textmd{7i.2}] System detects the conflict and prompts the Administrator to reload and retry.
                        \end{description}
                    \end{description}
                \end{quote}
                
			\item Third-Party Developers \#4
				\begin{quote}
				    N/A
				\end{quote}
                
			\item Environmental Monitoring Department \#5 \\
                \textbf{Pre-condition}: Department user is authenticated and has an assigned account.\\
                \textbf{Main Success Scenario}
                \begin{quote}
                    \begin{enumerate}[1.]
                        \item Department user opens their profile or account settings.
                        \item System displays the Department user’s assigned role and permitted actions.
                        \item Department user confirms what they can access within the system.
                    \end{enumerate}
                \end{quote}
                \textbf{Secondary Scenario}
                \begin{quote}
                    \begin{description}
                        \item[\textmd{3i.}] Needs access adjustment.
                        \begin{description}
                            \item[\textmd{3i.1}] Department user submits an access change request including the reason.
                            \item[\textmd{3i.2}] System records the request and notifies an Administrator for review.
                        \end{description}
                        \item[\textmd{2i.}] Role/permission information unavailable.
                        \begin{description}
                            \item[\textmd{2i.1}] System cannot load role/permission information.
                            \item[\textmd{2i.2}] System displays an error message and instructs the user to retry or contact IT Support.
                        \end{description}
                    \end{description}
                \end{quote}

			\item IT Support \#6 \\
                \textbf{Pre-condition}: IT Support is authenticated and authorized for support actions.\\
                \textbf{Main Success Scenario}
                \begin{quote}
                    \begin{enumerate}[1.]
                        \item IT Support opens an administrative support view for accounts and access.
                        \item System displays user status and role assignment information relevant to support.
                        \item IT Support performs an authorized support action (e.g., disable an account, restore access, or escalate).
                        \item System records the action in an audit/history log and notifies Administrators when required.
                    \end{enumerate}
                \end{quote}
                \textbf{Secondary Scenario}
                \begin{quote}
                    \begin{description}
                        \item[\textmd{3i.}] Emergency lockout required.
                        \begin{description}
                            \item[\textmd{3i.1}] IT Support detects suspicious activity or a security risk.
                            \item[\textmd{3i.2}] IT Support temporarily disables the affected account and escalates to an Administrator.
                        \end{description}
                        \item[\textmd{3ii.}] Insufficient authorization.
                        \begin{description}
                            \item[\textmd{3ii.1}] IT Support attempts an action outside of their permission scope.
                            \item[\textmd{3ii.2}] System blocks the action and recommends escalation to an Administrator.
                        \end{description}
                    \end{description}
                \end{quote}
		\end{enumerate}
        
        {\bf Global Scenario:}
        	\begin{quote}
        	    \textbf{Pre-condition}: At least one Administrator exists with authority to manage roles and permissions. \\
                \textbf{Main Success Scenario} 
                    \begin{quote}
                    \begin{enumerate}[1.]
                        \item A need arises to control or update what different users can do within the system.
                        \item An Administrator reviews existing roles, permissions, and user assignments.
                        \item The Administrator updates role permissions and/or user-role assignments as required.
                        \item The system validates changes, saves updates, and applies them to affected accounts.
                        \item The system records the changes in an audit/history log and notifies affected users if applicable.
                    \end{enumerate}
                    \end{quote}
                \textbf{Secondary Scenario} 
                \begin{quote}
                    \begin{description}
                        \item[\textmd{3i.}] Access change request submitted.
                        \begin{description}
                            \item[\textmd{3i.1}] A City Operator or Department user submits a request for additional or reduced access.
                            \item[\textmd{3i.2}] The system records the request and routes it to an Administrator for review and action.
                        \end{description}
                        \item[\textmd{4i.}] Security incident.
                        \begin{description}
                            \item[\textmd{4i.1}] IT Support detects suspicious behavior requiring immediate containment.
                            \item[\textmd{4i.2}] IT Support performs an authorized emergency action and escalates to an Administrator; the system logs the action.
                        \end{description}
                    \end{description}
                \end{quote}
        	\end{quote}
\end{enumerate}

\begin{enumerate}[{\bf BE5.}]
	\item City Operator Resolves an Alert \#5 \\      
    \textbf{Pre-condition}: An alert has been generated by the system and has been acknowledged by a city operator. The city operator is authenticated and logged into the system.
    
		\begin{enumerate}[{\bf VP1.}]
			\item City Operators \#1 
                \begin{quote}
                    N/A
                \end{quote}
                
			\item Public Users \#2 
				\begin{quote}
				    N/A
				\end{quote}
                
			\item Administrators \#3 \\
				\textbf{Main Success Scenario} 
                    \begin{quote}
                    \begin{enumerate}[1.]
                        \item System displays the dashboard showing the active alerts.
                        \item City operator selects an alert.
                        \item System displays the alert details.
                        \item City operator marks the alert as resolved.
                        \item System updates the alert status to \textit{Resolved}.
                        \item System records the time and the operators details.
                        \item System moves the alert to the \textit{Completed} page.
                        \item System confirms that the alert has been resolved.
                    \end{enumerate}
                    \end{quote}
                \textbf{Secondary Scenario} 
                \begin{quote}
                    \begin{description}
                        \item[\textmd{4i.}] Alert got resolved by another operator
                        \begin{description}
                            \item[\textmd{4i.1}] System refreshes the alert status.
                            \item[\textmd{4i.2}] System informs the city operator that the alert is already acknowledged.
                        \end{description}
                        \item[\textmd{5i.}] System fails to update the alert status.
                        \begin{description}
                            \item[\textmd{5i.1}] System displays an error message to city operator.
                            \item[\textmd{5i.2}] Alert remains in \textit{Acknowledged} status.
                            \item[\textmd{5i.2}] System provides instructions for contacting IT Support.
                        \end{description}
                    \end{description}
                \end{quote}
                
			\item Third-Party Developers \#4
				\begin{quote}
				    N/A
				\end{quote}
                
			\item Environmental Monitoring Department \#5
				\begin{quote}
				    N/A
				\end{quote}

			\item IT Support \#6
                \item[\textmd{5i}] System provides IT Support with system logs to investigate and resolve alert resolution failures or audit logging issues.
		\end{enumerate}
        
        {\bf Global Scenario:}
        	\begin{quote}
        	    \textbf{Pre-condition}: An alert has been generated by the system and has been acknowledged by a city operator. The city operator is authenticated and logged into the system. \\
                \textbf{Main Success Scenario} 
                    \begin{quote}
                    \begin{enumerate}[1.]
                        \item System displays the dashboard showing the active alerts.
                        \item City operator selects an alert.
                        \item System displays the alert details.
                        \item City operator marks the alert as resolved.
                        \item System updates the alert status to \textit{Resolved}.
                        \item System records the time and the operators details.
                        \item System moves the alert to the \textit{Completed} page.
                        \item System confirms that the alert has been resolved.
                    \end{enumerate}
                    \end{quote}
                \textbf{Secondary Scenario} 
                \begin{quote}
                    \begin{description}
                        \item[\textmd{4i.}] Alert got resolved by another operator
                        \begin{description}
                            \item[\textmd{4i.1}] System refreshes the alert status.
                            \item[\textmd{4i.2}] System informs the city operator that the alert is already acknowledged.
                        \end{description}
                        \item[\textmd{5i.}] System fails to update the alert status.
                        \begin{description}
                            \item[\textmd{5i.1}] System displays an error message to city operator.
                            \item[\textmd{5i.2}] Alert remains in \textit{Acknowledged} status.
                            \item[\textmd{5i.2}] System provides instructions for contacting IT Support.
                        \end{description}
                    \end{description}
                \end{quote}
        	\end{quote}
\end{enumerate}

\begin{enumerate}[{\bf BE6.}]
	\item Request Access to Public API
		\begin{enumerate}[{\bf VP1.}]
			\item City Operators \\
				    N/A. City Operators have access to environmental data internally and do not require public API credentials.

			\item Public Users \\
                \textbf{Pre-condition}: User has a registered Public User SCHEMAS account and is logged in. \\
                \textbf{Main Success Scenario}
                \begin{quote}
                    \begin{enumerate}[1.]
                        \item User selects ``Request Public API''.
                        \item System redirects user to an information page.
                        \item System informs user that public API access is unavailable and that an account modification to a Third-Party Developer account is required.
                    \end{enumerate}
                \end{quote}

			\item Administrators \\
                \textbf{Pre-condition}: User has a registered Administrator SCHEMAS account and is logged in. \\
                \textbf{Main Success Scenario}
                \begin{quote}
                    \begin{enumerate}[1.]
                        \item Administrator navigates to the admin dashboard.
                        \item Administrator selects ``View API request tickets''.
                        \item Administrator selects a ticket to view request details.
                        \item Administrator adds review comments.
                        \item Administrator approves the API access request.
                        \item System generates API credentials with appropriate rate limits.
                        \item System notifies the user that API access has been granted.
                    \end{enumerate}
                \end{quote}
                \textbf{Secondary Scenario}
                \begin{quote}
                    \begin{description}
                        \item[\textmd{5i.}] Administrator rejects API access request.
                        \begin{description}
                            \item[\textmd{5i.1}] System records the rejected request.
                            \item[\textmd{5i.2}] System notifies the user that the request has been rejected.
                        \end{description}
                    \end{description}
                \end{quote}

			\item Third-Party Developers \\
                \textbf{Pre-condition}: User has a registered Third-Party Developer SCHEMAS account and is logged in. \\
                \textbf{Main Success Scenario}
                \begin{quote}
                    \begin{enumerate}[1.]
                        \item User selects ``Request Public API''.
                        \item System displays the API access request form.
                        \item User enters required information including project name, purpose, organization affiliation, expected request volume, and required endpoints.
                        \item User submits the API access request form.
                        \item System validates the submitted information.
                        \item System stores the API access request record.
                        \item User receives notification of approved API access.
                        \item User opens the notification.
                        \item System displays API credentials and a link to API documentation.
                    \end{enumerate}
                \end{quote}
                \textbf{Secondary Scenario}
                \begin{quote}
                    \begin{description}
                        \item[\textmd{4i.}] System fails to create API access request.
                        \begin{description}
                            \item[\textmd{4i.1}] System displays a link to contact IT support.
                            \item[\textmd{4i.2}] User navigates to the IT support ticket page.
                            \item[\textmd{4i.3}] User creates a support ticket describing the issue.
                            \item[\textmd{4i.4}] User submits the support ticket.
                            \item[\textmd{4i.5}] IT Support responds with remediation instructions.
                            \item[\textmd{4i.6}] User follows instructions and returns to BE6-3.
                        \end{description}
                        \item[\textmd{6i.}] User receives notification that API access request was rejected.
                        \begin{description}
                            \item[\textmd{6i.1}] User opens the rejection notification.
                            \item[\textmd{6i.2}] System displays rejection reason and administrator comments.
                            \item[\textmd{6i.3}] User reviews comments and returns to BE6-1.
                        \end{description}
                    \end{description}
                \end{quote}

			\item Environmental Monitoring Department \\
				    N/A.
			\item IT Support \\
                \textbf{Pre-condition}: User has a registered IT Support SCHEMAS account and is logged in. \\
                \textbf{Main Success Scenario}
                \begin{quote}
                    \begin{enumerate}[1.]
                        \item IT Support navigates to the IT support ticket dashboard.
                        \item IT Support selects a support ticket to view details.
                        \item IT Support resolves the issue and documents actions taken.
                        \item IT Support marks the ticket as resolved.
                    \end{enumerate}
                \end{quote}
		\end{enumerate}

        {\bf Global Scenario:}
        \begin{quote}
            \textbf{Pre-condition}: User has a registered Third-Party Developer SCHEMAS account and is logged in. \\
            \textbf{Main Success Scenario}
            \begin{quote}
                \begin{enumerate}[1.]
                    \item User selects ``Request Public API''.
                    \item System displays the API access request form.
                    \item User enters required project and usage information.
                    \item User submits the API access request form.
                    \item System validates and stores the request.
                    \item Administrator reviews and approves the request.
                    \item System generates API credentials with appropriate rate limits.
                    \item System notifies the user that API access has been granted.
                \end{enumerate}
            \end{quote}
            \textbf{Secondary Scenario}
            \begin{quote}
                \begin{description}
                    \item[\textmd{4i.}] System fails to submit API access request form.
                    \begin{description}
                        \item[\textmd{4i.1}] System prompts user to contact IT support.
                        \item[\textmd{4i.2}] User submits an IT support ticket.
                        \item[\textmd{4i.3}] IT Support resolves the issue.
                        \item[\textmd{4i.4}] User retries the API request.
                    \end{description}
                    \item[\textmd{5i.}] System detects invalid form fields.
                    \begin{description}
                        \item[\textmd{5i.1}] User is prompted to correct the form.
                        \item[\textmd{5i.2}] User resubmits the request.
                    \end{description}
                    \item[\textmd{6i.}] Administrator rejects the request.
                    \begin{description}
                        \item[\textmd{6i.1}] User receives rejection notification.
                        \item[\textmd{6i.2}] User reviews comments and returns to BE6-1.
                    \end{description}
                \end{description}
            \end{quote}
        \end{quote}
\end{enumerate}

\begin{enumerate}[{\bf BE7.}]
	\item Modify Account
		\begin{enumerate}[{\bf VP1.}]
			\item City Operators \\
				    N/A. City Operators do not have permission to modify user accounts.

			\item Public Users \\
                \textbf{Pre-condition}: User has a valid Public User SCHEMAS account and is logged in. \\
                \textbf{Main Success Scenario}
                \begin{quote}
                    \begin{enumerate}[1.]
                        \item User selects ``Request Account Modification''.
                        \item System displays an account modification page containing a request form.
                        \item User enters the purpose of the request, organization affiliation, and developer credentials (if required).
                        \item User submits the account modification request form.
                        \item System records the account modification request.
                        \item System notifies the user of the account modification decision.
                        \item User opens the notification.
                        \item System displays confirmation of the change, updated permission details, and administrator comments.
                    \end{enumerate}
                \end{quote}
                \textbf{Secondary Scenario}
                \begin{quote}
                    \begin{description}
                        \item[\textmd{3i.}] User submits invalid or incomplete form details.
                        \begin{description}
                            \item[\textmd{3i.1}] System prompts the user to correct the form and retry submission.
                        \end{description}
                        \item[\textmd{4i.}] System prompts user to contact customer support.
                        \begin{description}
                            \item[\textmd{4i.1}] User navigates to the customer support ticket page.
                            \item[\textmd{4i.2}] User submits a support ticket describing the issue.
                            \item[\textmd{4i.3}] IT Support responds with remediation instructions.
                            \item[\textmd{4i.4}] User follows instructions and returns to BE7-1.
                        \end{description}
                        \item[\textmd{6i.}] Administrator rejects the account modification request.
                        \begin{description}
                            \item[\textmd{6i.1}] User receives a rejection notification.
                            \item[\textmd{6i.2}] User reviews comments and returns to BE7-1.
                        \end{description}
                    \end{description}
                \end{quote}

			\item Administrators \\
                \textbf{Pre-condition}: User has a valid Administrator SCHEMAS account and is logged in. \\
                \textbf{Main Success Scenario}
                \begin{quote}
                    \begin{enumerate}[1.]
                        \item Administrator navigates to the admin dashboard.
                        \item Administrator selects ``View Account Modification Request Tickets''.
                        \item Administrator selects a ticket to view request details.
                        \item Administrator adds review comments.
                        \item Administrator approves the account modification request.
                    \end{enumerate}
                \end{quote}

			\item Third-Party Developers \\
                \textbf{Pre-condition}: User has a valid Third-Party Developer SCHEMAS account and is logged in. \\
                \textbf{Main Success Scenario}
                \begin{quote}
                    \begin{enumerate}[1.]
                        \item Developer selects ``Request Account Modification''.
                        \item System displays an account modification page containing a request form.
                        \item Developer enters the purpose of the request, organization affiliation, and developer credentials (if required).
                        \item Developer submits the account modification request form.
                        \item System records the account modification request.
                        \item System notifies the developer of the account modification decision.
                        \item Developer opens the notification.
                        \item System displays confirmation of the change, updated permission details, and administrator comments.
                    \end{enumerate}
                \end{quote}
                \textbf{Secondary Scenario}
                \begin{quote}
                    \begin{description}
                        \item[\textmd{3i.}] Developer submits invalid or incomplete form details.
                        \begin{description}
                            \item[\textmd{3i.1}] System prompts the developer to correct the form and retry submission.
                        \end{description}
                        \item[\textmd{4i.}] System prompts developer to contact customer support.
                        \begin{description}
                            \item[\textmd{4i.1}] Developer navigates to the customer support ticket page.
                            \item[\textmd{4i.2}] Developer submits a support ticket describing the issue.
                            \item[\textmd{4i.3}] IT Support responds with remediation instructions.
                            \item[\textmd{4i.4}] Developer follows instructions and returns to BE7-1.
                        \end{description}
                        \item[\textmd{6i.}] Administrator rejects the account modification request.
                        \begin{description}
                            \item[\textmd{6i.1}] Developer receives a rejection notification.
                            \item[\textmd{6i.2}] Developer reviews comments and returns to BE7-1.
                        \end{description}
                    \end{description}
                \end{quote}

			\item Environmental Monitoring Department \\
				    N/A.

			\item IT Support \\
                \textbf{Pre-condition}: User has a registered IT Support SCHEMAS account and is logged in. \\
                \textbf{Main Success Scenario}
                \begin{quote}
                    \begin{enumerate}[1.]
                        \item IT Support navigates to the IT support ticket dashboard.
                        \item IT Support selects a support ticket to view details.
                        \item IT Support resolves the issue and documents actions taken.
                    \end{enumerate}
                \end{quote}
		\end{enumerate}

        {\bf Global Scenario:}
        \begin{quote}
            \textbf{Pre-condition}: User has a valid Public User or Third-Party Developer SCHEMAS account and is logged in. \\
            \textbf{Main Success Scenario}
            \begin{quote}
                \begin{enumerate}[1.]
                    \item User selects ``Request Account Modification''.
                    \item System displays the account modification request form.
                    \item User submits required modification details.
                    \item Administrator reviews the modification request.
                    \item Administrator approves the request.
                    \item System records the account modification.
                    \item System notifies the user of the modification.
                    \item User reviews confirmation and updated permissions.
                \end{enumerate}
            \end{quote}
            \textbf{Secondary Scenario}
            \begin{quote}
                \begin{description}
                    \item[\textmd{3i.}] User submits invalid or incomplete form details.
                    \begin{description}
                        \item[\textmd{3i.1}] User is prompted to retry submission.
                    \end{description}
                    \item[\textmd{4i.}] System prompts user to contact customer support.
                    \begin{description}
                        \item[\textmd{4i.1}] User submits a support ticket.
                        \item[\textmd{4i.2}] IT Support resolves the issue.
                        \item[\textmd{4i.3}] User retries the modification request.
                    \end{description}
                    \item[\textmd{9i.}] Administrator rejects the request.
                    \begin{description}
                        \item[\textmd{9i.1}] User receives rejection notification.
                        \item[\textmd{9i.2}] User reviews comments and returns to BE7-1.
                    \end{description}
                \end{description}
            \end{quote}
        \end{quote}
\end{enumerate}



%	Below, we organize by Business Event.
%	\begin{enumerate}[{BE}1.]
%		\item Business Event name
%		\begin{enumerate}[{VP1}.1]
%			\item Viewpoint name \newline
%			\noindent\fbox{%
%				\parbox{0.5\textwidth}{%
%					\begin{itemize}
%						\item {\bf $S_{1}$:} Initial response of the system to the Business Event
%						\item {\bf $E_{1}$:}  Reaction of the environment to $S_{1}$
%						\item {\bf $S_{2}$:}  Response of the system to $E_{1}$
%						\item {\bf $E_{2}$:}  Reaction of the environment to $S_{2}$
%						\item[] $\cdots$
%						\item {\bf $S_{n}$:}  Response of the system to $E_{(n-1)}$
%						\item {\bf $E_{n}$:}  Reaction of the environment to $E_{(n-1)}$
%						\item {\bf $S_{(n+1)}$:} Final response of the system concluding its function regarding the Business Event
%					\end{itemize}
%				}%
%			}
%			\item Viewpoint name\newline
%			\noindent\fbox{%
%				\parbox{0.5\textwidth}{%
%					\begin{itemize}
%						\item {\bf $S_{1}$:} Initial response of the system to the Business Event
%						\item {\bf $E_{1}$:}  Reaction of the environment to $S_{1}$
%						\item {\bf $S_{2}$:}  Response of the system to $E_{1}$
%						\item {\bf $E_{2}$:}  Reaction of the environment to $S_{2}$
%						\item[] $\cdots$
%						\item {\bf $S_{k}$:}  Response of the system to $E_{(k-1)}$
%						\item {\bf $E_{k}$:}  Reaction of the environment to $E_{(k-1)}$
%						\item {\bf $S_{(k+1)}$:} Final response of the system concluding its function regarding the Business Event
%					\end{itemize}
%				}%
%			}
%			\item \dots
%			\item \dots
%			\item \dots
%			\item[\dots]
%		\end{enumerate}	
%		\item[] {\bf Global Scenario of {\it Business Event Name}:} It is the scenario corresponding to the integration of all the above scenarios from the different Viewpoints of the Business Event BE1.\newline
%		\noindent\fbox{%
%			\parbox{0.5\textwidth}{%
%				\begin{itemize}
%					\item {\bf $S_{1}$:} Initial response of the system to the Business Event
%					\item {\bf $E_{1}$:}  Reaction of the environment to $S_{1}$
%					\item {\bf $S_{2}$:}  Response of the system to $E_{1}$
%					\item {\bf $E_{2}$:}  Reaction of the environment to $S_{2}$
%					\item[] $\cdots$
%					\item {\bf $S_{m}$:}  Response of the system to $E_{(m-1)}$
%					\item {\bf $E_{m}$:}  Reaction of the environment to $E_{(m-1)}$
%					\item {\bf $S_{(m+1)}$:} Final response of the system concluding its function regarding the Business Event
%				\end{itemize}
%			}%
%		}	
%		%\end{enumerate}
%		\item Business Event name
%		\begin{enumerate}[{VP1}.1]
%			\item Viewpoint name \newline
%			\noindent\fbox{%
%				\parbox{0.5\textwidth}{%
%					\begin{itemize}
%						\item {\bf $S_{1}$:} Initial response of the system to the Business Event
%						\item {\bf $E_{1}$:}  Reaction of the environment to $S_{1}$
%						\item {\bf $S_{2}$:}  Response of the system to $E_{1}$
%						\item {\bf $E_{2}$:}  Reaction of the environment to $S_{2}$
%						\item[] $\cdots$
%						\item {\bf $S_{n'}$:}  Response of the system to $E_{(n'-1)}$
%						\item {\bf $E_{n'}$:}  Reaction of the environment to $E_{(n'-1)}$
%						\item {\bf $S_{(n'+1)}$:} Final response of the system concluding its function regarding the Business Event
%					\end{itemize}
%				}%
%			}
%			\item Viewpoint name\newline
%			\noindent\fbox{%
%				\parbox{0.5\textwidth}{%
%					\begin{itemize}
%						\item {\bf $S_{1}$:} Initial response of the system to the Business Event
%						\item {\bf $E_{1}$:}  Reaction of the environment to $S_{1}$
%						\item {\bf $S_{2}$:}  Response of the system to $E_{1}$
%						\item {\bf $E_{2}$:}  Reaction of the environment to $S_{2}$
%						\item[] $\cdots$
%						\item {\bf $S_{k'}$:}  Response of the system to $E_{(k'-1)}$
%						\item {\bf $E_{k'}$:}  Reaction of the environment to $E_{(k'-1)}$
%						\item {\bf $S_{(k'+1)}$:} Final response of the system concluding its function regarding the Business Event
%					\end{itemize}
%				}%
%			}
%			\item \dots
%			\item \dots
%			\item \dots
%			\item[\dots]
%		\end{enumerate}	
%		\item[] {\bf Global Scenario of {\it Business Event Name}:} It is the scenario corresponding to the integration of all the above scenarios from the different Viewpoints of the Business Event BE2.\newline
%		\noindent\fbox{%
%			\parbox{0.5\textwidth}{%
%				\begin{itemize}
%					\item {\bf $S_{1}$:} Initial response of the system to the Business Event
%					\item {\bf $E_{1}$:}  Reaction of the environment to $S_{1}$
%					\item {\bf $S_{2}$:}  Response of the system to $E_{1}$
%					\item {\bf $E_{2}$:}  Reaction of the environment to $S_{2}$
%					\item[] $\cdots$
%					\item {\bf $S_{m'}$:}  Response of the system to $E_{(m'-1)}$
%					\item {\bf $E_{m'}$:}  Reaction of the environment to $E_{(m'-1)}$
%					\item {\bf $S_{(m'+1)}$:} Final response of the system concluding its function regarding the Business Event
%				\end{itemize}
%			}%
%		}		
%	\end{enumerate}

%End Section

\section{Non-Functional Requirements}
\label{sec:non-functional_requirements}


\begin{itemize}
	\item For each non-functional requirement, provide a justification/rationale for it.\\
	{\bf Example:} \\
	SC1. \emph{The device should not explode in a customer’s pocket.}\\
	{\bf Rationale:} Other companies have had issues with the batteries they used in their phones randomly exploding [insert citation]. This causes a safety issue, as the phone is often carried in a person's hand or pocket.	
	\item If you need to make a guess because you couldn't really talk to stakeholders, you can say "We imagined stakeholders would want...because..."
	\item Each requirement should have a unique label/number for it.
	\item In the list below, if a particular section doesn't apply, just write N/A so we know you considered it.
\end{itemize}

% Begin Section
\subsection{Look and Feel Requirements}
\label{sub:look_and_feel_requirements}
% Begin SubSection

\subsubsection{Appearance Requirements}
\label{ssub:appearance_requirements}
% Begin SubSubSection
\begin{enumerate}[{LF-A}1. ]
	\item The system shall use a cool, non-distracting background that does not draw visual focus away from the operational dashboard content.\\
	\textbf{Rationale:} SCEMAS is intended for monitoring environmental conditions. A background with colours that are not warm such as bright red/yellow ensures that users’ attention is reserved for critical dashboard elements such as metrics, dashboards, and alerts.
	\item The system should use a consistent font family across all screens.\\
	\textbf{Rationale:} A standardized font enhances the user experience by providing a consistent experience across all screens making it easier to navigate.
	\item Visual alert indicators shall be designed to clearly stand out from non-critical information.\\
	\textbf{Rationale:} Environmental alerts often require immediate action. Distinct visual indicators ensure that critical conditions are quickly recognizable, even when operators are monitoring large volumes of data.
	\item The system shall use a font size that is readable; the font size should also be consistent across all screens relative to the header type.\\
	\textbf{Rationale:} This ensures that information can be categorized into classes such as title, heading, paragraph heading and paragraph body while being readable to users.
\end{enumerate}
% End SubSubSection

\subsubsection{Style Requirements}
\label{ssub:style_requirements}
% Begin SubSubSection
\begin{enumerate}[{LF-S}1. ]
	\item The system shall adapt its layout to different screen resolutions.\\
	\textbf{Rationale:} City operators may access the system from devices with varying screen sizes. Responsive layout behavior ensures a consistent experience across supported devices.
	\item The system shall use a consistent layout that is the same for all dashboard views.\\
	\textbf{Rationale:} A consistent layout structure helps users build familiarity with the interface, preventing unnecessary navigation or relearning on every new screen.
	\item The system shall group related information and controls using spacing, alignment, and section boundaries.\\
	\textbf{Rationale:} Clear visual grouping improves information organization and allows users to quickly identify related data.
	\item The system shall limit the text density on each screen.\\
	\textbf{Rationale:} Screens should be informative and require as little navigation as possible but not overwhelm the user with information cluttering the user experience.
\end{enumerate}
% End SubSubSection

% End SubSection


\subsection{Usability and Humanity Requirements}
\label{sub:usability_and_humanity_requirements}
% Begin SubSection

\subsubsection{Ease of Use Requirements}
\label{ssub:ease_of_use_requirements}
% Begin SubSubSection
\begin{enumerate}[{UH-EOU}1. ]
	\item The system should show brief tooltips for all major controls, indicators and buttons.\\
	\textbf{Rationale:} Giving tooltips enhances user understanding and provides context for each element on the dashboard.
	\item The system should allow users to report bugs to IT support.\\
	\textbf{Rationale:} If users encounter a problem they should be able to receive a response from the SCHEMAS team to address their issue and implement fixes.
	\item The system should maintain consistent terminology and terms throughout the interface.\\
	\textbf{Rationale:} Consistent terminology improves learnability and prevents inconsistent descriptions across multiple screens.
	\item The system shall cache user interface preferences, such as selected dashboards or view settings.\\
	\textbf{Rationale:} A cache on the preference allows users to have a consistent configuration in which they view the dashboard which makes the screens tailor made for any particular user.
	\item The system shall contain a Frequently Asked Question section.\\
	\textbf{Rationale:} An FAQ section allows users to get quick and easy answers to commonly asked questions reducing the time required to get acclimated to the system [1] (https://masterful-marketing.com/website-faq-benefits-best-practices/).
\end{enumerate}
% End SubSubSection

\subsubsection{Personalization and Internationalization Requirements}
\label{ssub:personalization_and_internationalization_requirements}
% Begin SubSubSection
\begin{enumerate}[{UH-PI}1. ]
	\item The system shall allow users to configure personal interface preferences.\\
	\textbf{Rationale:} Allowing users to change their default views, dashboard preference and interface preferences will help increase usability among users.
	\item The system shall allow users to change their language preferences.\\
	\textbf{Rationale:} All users should be able to understand the information on the screen. This is essential for all users to be able to view the system equally and accurately.
	\item The system shall support internationalized measurement units, date formats and time zones.\\
	\textbf{Rationale:} Environmental data must be able to be measured and accessed across all time zones. Supporting international viewing standards ensure data is accessible to people regardless of their location.
\end{enumerate}
% End SubSubSection

\subsubsection{Learning Requirements}
\label{ssub:learning_requirements}
% Begin SubSubSection
\begin{enumerate}[{UH-L}1. ]
	\item Users shall be able to learn how to navigate the main dashboard and interpret key environmental metrics within 20 minutes of first use.\\
	\textbf{Rationale:} Operators must quickly understand the core interface to respond to environmental events effectively. Limiting the learning time ensures the system is intuitive and reduces onboarding effort.
	\item Users shall be able to configure personal preferences, such as measurement units, default dashboards, and theme settings, within 10 minutes of first use.\\
	\textbf{Rationale:} Users should quickly learn how best to tailor the application to them. If they are able to personalize the system quickly it will make sure that the rest of their time using the system is more enjoyable.
    \item A trained city operator(someone that has 10+ hours of education on the system) must be able
    to log into the dashboard and successfully acknowledge a critical environmental alert within 30 seconds of the
    alert appearing.\\
	\textbf{Rationale:} City operators that are familiar with the system should be able to navigate the dashboard and acknowledge an alert, this shows the system is user-friendly and accessible.
    \item A trained developer (someone that has 4+ years of experience developing) must be able to understand the API documentation within a 1 hour time frame.\\
	\textbf{Rationale:} Developers that are willing to subscribe to APIs should be able to understand the documentation in the short(1-hour) time frame. If they are able to understand the documentation within this time it means that the documentation is written clearly and helps improve usability of the service.
\end{enumerate}
% End SubSubSection

\subsubsection{Understandability and Politeness Requirements}
\label{ssub:understandability_and_politeness_requirements}
% Begin SubSubSection
\begin{enumerate}[{UH-UP}1. ]
	\item The system shall use professional and neutral wording in all user-facing communications.\\
	\textbf{Rationale:} Maintaining a professional tone is important to ensure users are not offended by any terminology and allows users to strictly focus on the data.
	\item The system shall avoid technical jargon or abbreviations in messages unless universally understood by target users.\\
	\textbf{Rationale:} Using understandable and general knowledge ensures users of all levels of technicality can understand the information [2] https://www.nngroup.com/articles/technical-jargon/. This prevents confusion and ensures understandability.
\end{enumerate}
% End SubSubSection

\subsubsection{Accessibility Requirements}
\label{ssub:accessibility_requirements}
% Begin SubSubSection
\begin{enumerate}[{UH-A}1. ]
	\item The system shall provide text-to-speech functionality for all essential interface elements.\\
	\textbf{Rationale:} Text to speech allows users that are blind or low-vision to interpret environmental data.
	\item The system shall provide captions or transcripts for all audio content, including notifications and messages.\\
	\textbf{Rationale:} Users who are deaf or hard-of-hearing should be able to  access spoken information like any other able user and a viable way to do this is using text-to-speech [3]. https://voxpow.com/blog/what-is-text-to-speech/.
	\item The system shall use a color palette that is accessible to users with common forms of color vision deficiency.\\
	\textbf{Rationale:} City-operated systems must be inclusive and accessible. 1/16 men and 1/240 women have some form of colour deficiency and therefore should be accommodated [4] \texttt{https://davidmathlogic.com/colorblind/\#D81B60-\#1E88E5-\#FFC107-\#004D40}. Colorblind-friendly design ensures that all operators can correctly interpret system states and alerts without relying solely on colour perception.
\end{enumerate}
% End SubSubSection

% End SubSection


\subsection{Performance Requirements}
\label{sub:performance_requirements}
% Begin SubSection

\subsubsection{Speed and Latency Requirements}
\label{ssub:speed_and_latency_requirements}
% Begin SubSubSection
\begin{enumerate}[{PR-SL}1.]
	\item The web dashboard for city operators should have a load time of less than 1 second\\
    \textbf{Rationale}: Having a fast-loading website will ensure users remain engaged. A study shows that 40\% of users will leave a website if it takes longer than 3 seconds to load [https://www.browserstack.com/guide/how-fast-should-a-website-load].
    \item The system should provide a REST API for public use with a response time of less than 1 second\\
    \textbf{Rationale}: APIs with lower response time lead to many benefits, like enhancing user satisfaction and increasing cost effectiveness of the system by allowing fewer resources to be used. tied up in operations [https://odown.com/blog/what-is-a-good-api-response-time/#why-api-response-time-matters]. longer API response times will likely result in user frustration and abandonment[https://odown.com/blog/what-is-a-good-api-response-time/#why-api-response-time-matters].
\end{enumerate}
% End SubSubSection

\subsubsection{Safety-Critical Requirements}
\label{ssub:safety_critical_requirements}
% Begin SubSubSection
\begin{enumerate}[{PR-SC}1. ]
	\item The system should ensure that alerts are posted in less than 5 seconds of the internal system finding a threshold broken.\\
    \textbf{Rationale}: A rapid alert system allows operations to respond to any alerts within a timely manner, staging interventions or putting out city wide alerts where possible. 
\end{enumerate}
% End SubSubSection

\subsubsection{Precision or Accuracy Requirements}
\label{ssub:precision_or_accuracy_requirements}
% Begin SubSubSection
\begin{enumerate}[{PR-PA}1. ]
	\item For each environmental indicator (air quality, noise levels, temperature, humidity) the system shall have a standard deviation of less than 15\%.\\
    \textbf{Rationale}: Inaccurate data could lead to a false sense of security or a false sense of urgency when trying to address the implications of the measurement[https://pollution.sustainability-directory.com/term/environmental-monitoring-precision/], its important for the measurement to remain accurate to avoid these situations.
\end{enumerate}
% End SubSubSection

\subsubsection{Reliability and Availability Requirements}
\label{ssub:reliability_and_availability_requirements}
% Begin SubSubSection
\begin{enumerate}[{PR-RA}1. ]
	\item The database should have an up time of at least 99\%\\
    \textbf{Rationale}: This allows any authentication requests to properly go through and avoids cases where users cannot use the system due to authentication requests.
    \item The system should ensure that alerts have a false positive rating of less than 1\%\\
    \textbf{Rationale}: A high false positive rating can lead to improper use of resources, impaired efficiency and a loss of trust of users [https://vwo.com/glossary/false-positive-rate/]. Its important to have a low false positive rate to avoid taking unnecessary measures to correct a false issue.
\end{enumerate}
% End SubSubSection

\subsubsection{Robustness or Fault-Tolerance Requirements}
\label{ssub:robustness_or_fault_tolerance_requirements}
% Begin SubSubSection
\begin{enumerate}[{PR-RFT}1. ]
	\item The System should not fail if one component of it fails.\\
    \textbf{Rationale}: This is to ensure that the system stays functional even if a sensor fails, meaning the system will continue to collect data from the other sensors.
    \item The system should detect and log any errors in under 5 seconds\\
    \textbf{Rationale}: This is to ensure that all errors are properly logged as well as properly addressed within a timely manner.
\end{enumerate}
% End SubSubSection

\subsubsection{Capacity Requirements}
\label{ssub:capacity_requirements}
% Begin SubSubSection
\begin{enumerate}[{PR-C}1. ]
	\item The system should be able handle and store information from at least 150 sensors across the city.\\
    \textbf{Rationale}: A mid sized city requires 150 sensors[https://pmc.ncbi.nlm.nih.gov/articles/PMC8991303/] to allow the system to collect accurate data for accurate measurements. Ensuring the whole city is covered allows us to take action where its needed.
\end{enumerate}
% End SubSubSection

\subsubsection{Scalability or Extensibility Requirements}
\label{ssub:scalability_or_extensibility_requirements}
% Begin SubSubSection
\begin{enumerate}[{PR-SE}1. ]
	\item The internal code of our system should follow SOLID design principles to ensure scalability.\\ 
    \textbf{Rationale}: The Solid design principles are a set of principles that help developers write clean, maintainable, and scalable code [https://medium.com/@ucgorai/the-solid-approach-principles-for-maintainable-and-scalable-code-1208473e1bf2]. Adopting these principles in the system's design allows our system to be scalable and extensible. 
\end{enumerate}
% End SubSubSection

\subsubsection{Longevity Requirements}
\label{ssub:longevity_requirements}
% Begin SubSubSection
\begin{enumerate}[{PR-L}1. ]
	\item The sensors should use sustainable materials that guarantee a lifespan of at least 2 years\\
    \textbf{Rationale}: This is to ensure that each sensor can work optimally without the need to constantly replace them to ensure proper data collection.
\end{enumerate}
% End SubSubSection

% End SubSection

\subsection{Operational and Environmental Requirements}
\label{sub:operational_and_environmental_requirements}
% Begin SubSection

\subsubsection{Expected Physical Environment}
\label{ssub:expected_physical_environment}
% Begin SubSubSection
\begin{enumerate}[{OE-EPE}1. ]
	\item SCEMAS shall operate in a cloud-hosted environment using commodity server infrastructure (virtual machines/containers) and shall not require any specialized physical hardware.\\
    \textbf{Rationale:} The system should be deployable using free-tier or educational cloud resources, and sensor hardware is simulated rather than physically deployed.
    
	\item SCEMAS shall support operation over standard internet connections and tolerate intermittent connectivity from simulated sensor publishers by continuing to run and ingest telemetry once connectivity resumes.\\
    \textbf{Rationale:} IoT-style telemetry publishers may disconnect; the system should remain stable and continue ingesting when publishers reconnect.
\end{enumerate}
% End SubSubSection

\subsubsection{Requirements for Interfacing with Adjacent Systems}
\label{ssub:requirements_for_interfacing_with_adjacent_systems}
% Begin SubSubSection
\begin{enumerate}[{OE-IA}1. ]
	\item SCEMAS shall ingest sensor telemetry through the MQTT protocol and validate each incoming message against a defined schema and plausible value ranges before storage.\\
    \textbf{Rationale:} MQTT ingestion and validation are core requirements for reliable telemetry processing.
    
	\item SCEMAS shall expose a read-only REST API for public/third-party consumption that returns aggregated, non-sensitive environmental data.\\
    \textbf{Rationale:} The public-facing API must be read-only and provide aggregated data rather than sensitive/raw details.
    
	\item When an alert is triggered, SCEMAS shall notify subscribed external systems through a dedicated notification API endpoint.\\
    \textbf{Rationale:} External notifications on alert triggers are a required system capability.
\end{enumerate}
% End SubSubSection

\subsubsection{Productization Requirements}
\label{ssub:productization_requirements}
% Begin SubSubSection
\begin{enumerate}[{OE-P}1. ]
	\item SCEMAS shall provide configuration-driven deployment for thresholds, zone definitions, and alert rules so the same build can be used across environments (dev/test/demo).\\
    \textbf{Rationale:} Configuration-based behavior supports different setups without rebuilding code and better fits a reusable platform.
    
	\item SCEMAS shall include clear operator and developer documentation, including public API documentation sufficient for a competent third-party developer to request and interpret data within two hours.\\
    \textbf{Rationale:} Good documentation is required so external developers can successfully use the API quickly.
\end{enumerate}
% End SubSubSection

\subsubsection{Release Requirements}
\label{ssub:release_requirements}
% Begin SubSubSection
\begin{enumerate}[{OE-R}1. ]
	\item SCEMAS shall support separate deployment profiles for (at minimum) development and demonstration/staging, with distinct configuration for database endpoints, authentication secrets, and rate limits.\\
    \textbf{Rationale:} Separating environments reduces demo risk and avoids mixing secrets/data across deployments.
    
	\item Each SCEMAS release shall be versioned and include a rollback strategy for the dashboard and API services.\\
    \textbf{Rationale:} Cloud deployments can fail; rollback supports service continuity.
\end{enumerate}
% End SubSubSection

% End SubSection


\subsection{Maintainability and Support Requirements}
\label{sub:maintainability_and_support_requirements}
% Begin SubSection

\subsubsection{Maintenance Requirements}
\label{ssub:maintenance_requirements}
% Begin SubSubSection
\begin{enumerate}[{MS-M}1. ]
	\item SCEMAS shall be designed as modular services/components (e.g., ingestion, alerting, storage, dashboard/API) so that changes to one component do not require rewriting unrelated components.\\
    \textbf{Rationale:} Modularity reduces regression risk and makes future extensions (new metrics, new rules) easier.
    
	\item SCEMAS shall maintain automated logs for major runtime events (telemetry ingestion failures, validation rejections, alert triggers, and admin actions) to support debugging and ongoing maintenance.\\
    \textbf{Rationale:} Logs are essential for diagnosing issues in distributed systems and supporting ongoing maintenance.
\end{enumerate}
% End SubSubSection

\subsubsection{Supportability Requirements}
\label{ssub:supportability_requirements}
% Begin SubSubSection
\begin{enumerate}[{MS-S}1. ]
	\item SCEMAS shall provide a support-facing troubleshooting view or documented runbook steps that allow IT Support/Administrators to identify common failures (database unreachable, API rate-limit triggered) and the recommended recovery action.\\
    \textbf{Rationale:} Support needs a fast way to narrow down what broke in a multi-component system (ingestion + storage + dashboards + API).
    
	\item SCEMAS shall provide user-facing error messages on the dashboard/API that are actionable (what happened + what to try next) without exposing sensitive internal details.\\
    \textbf{Rationale:} Clear errors reduce support load, and avoiding internal details supports secure operations.
\end{enumerate}
% End SubSubSection

\subsubsection{Adaptability Requirements}
\label{ssub:adaptability_requirements}
% Begin SubSubSection
\begin{enumerate}[{MS-A}1. ]
	\item SCEMAS shall allow new sensor types/metrics to be introduced by extending the telemetry schema and validation rules, without changing the operator dashboard’s core navigation or authentication flow.\\
    \textbf{Rationale:} Monitoring evolves; the system should adapt to new indicators while keeping the operator experience stable.
    
	\item SCEMAS shall be adaptable to multiple deployment providers (or local containers for demos) by avoiding provider-specific hard dependencies in core logic.\\
    \textbf{Rationale:} Teams may switch hosting/deployment options; portability reduces rework.
\end{enumerate}
% End SubSubSection

% End SubSection

\subsection{Security Requirements}
\label{sub:security_requirements}
% Begin SubSection

\subsubsection{Access Requirements}
\label{ssub:access_requirements}
% Begin SubSubSection
\begin{enumerate}[{SR-AC}1. ]
	\item The system must require all users of the dashboard to authenticate using an email/username and password, being granted access to the system. \\
    \textbf{Rationale}: This ensures that only people who are authorized can access the system, which prevents the risk of authorized individuals to alter environmental data [https://www.cyber.gc.ca/en/guidance/user-authentication-guidance-information-technology-systems-itsp30031-v3].

    \item The system must support multi-factor authentication (MFA) for administrative accounts. \\
    \textbf{Rationale}: MFA helps by eliminating the possibility of compromising the credentials of an administrators account. This is crucial, since the administrator is a high-privilege user who can define alert rules, and view sensitive system information [https://www.cyber.gc.ca/en/guidance/secure-your-accounts-and-devices-multi-factor-authentication-itsap30030].

    \item The system must enforce Role-Base Access Control (RBAC) to distinguish between the different user groups, such as administrator, city operator, and read-only public users. \\
    \textbf{Rationale}: Enforcing an RBAC policy ensures that each user group can only perform actions that are appropriate to their respective role. This helps to reduce accidental misuse of the systems functions. 

    \item The system must lock out a user account after five consecutive failed login attempts. \\
    \textbf{Rationale}: Protects against repeated attempts of trying to login to an account that does not belong to the respective individual [https://www.cyber.gc.ca/en/guidance/user-authentication-guidance-information-technology-systems-itsp30031-v3].

    \item All IoT devices must be authenticated with a unique key before sending telemetry data to the system. \\
    \textbf{Rationale}: Prevents harmful telemetry to be injection into the system, by limiting entering data to registered and authorized sensors [https://www.cyber.gc.ca/en/guidance/internet-things-iot-security-itsap00012].
\end{enumerate}
% End SubSubSection

\subsubsection{Integrity Requirements}
\label{ssub:integrity_requirements}
% Begin SubSubSection
\begin{enumerate}[{SR-INT}1. ]
    \item Any data transmitted between IoT devices, and the system must be encrypted, and include integrity protection. \\
    \textbf{Rationale}: Ensures that telemetry is protected, and cannot be changed during transmission. As a result, this helps maintain confidentiality and integrity [https://www.cyber.gc.ca/en/guidance/using-encryption-keep-your-sensitive-data-secure-itsap40016].
    
	\item All incoming sensor telemetry must be validated against a defined schema and acceptable value ranges before being stored. \\
    \textbf{Rationale}: Ensures that only correct and meaningful data is allowed to enter the system. This restricts any corrupted/malicious data from affecting any environmental analytics and alerts. 

    \item The system must maintain an audit log to keep track of any data modification, such as alert rule updates, or alert status changes. \\
    \textbf{Rationale}: An audit log is essential to provide traceability to identify any potential modifications made accidentally, or by unauthorized personnel [https://www.researchgate.net/publication/352745109\_Audit\_Logs\_Management\_and\_Security\_-\_A\_Survey].

    \item The system should use cryptographic hashing for any alert generated, to ensure that the alert data remains secure, and unaltered after generation. \\
    \textbf{Rationale}: Enforces data integrity, and prevents city operators from receiving false notifications [https://www.cyber.gc.ca/en/guidance/guidance-securely-configuring-network-protocols-itsp40062].
\end{enumerate}
% End SubSubSection

\subsubsection{Privacy Requirements}
\label{ssub:privacy_requirements}
% Begin SubSubSection
\begin{enumerate}[{SR-P}1. ]
	\item The system will not collect or store any personally identifiable information from public users. \\
    \textbf{Rationale}: Protects individual privacy, and ensures compliance with PIPEDA regulations [https://www.priv.gc.ca/en/privacy-topics/privacy-laws-in-canada/the-personal-information-protection-and-electronic-documents-act-pipeda/p\_principle/principles/p\_collection/]

    \item The system shall only collect environmental data at the level of general city zones. \\
    \textbf{Rationale}: Ensures citizen privacy by preventing sensor data from tracking individual residents

    \item Public APIs should only provide non-sensitive data to public citizens and third-party developers. \\
    \textbf{Rationale}: Prevents misuse of the API, and protects privacy.
    
\end{enumerate}
% End SubSubSection

\subsubsection{Audit Requirements}
\label{ssub:audit_requirements}
% Begin SubSubSection
\begin{enumerate}[{SR-AU}1. ]
	\item The system shall log all user login attempts, including the timestamp, their distinct identification number, and the location of the request. \\
    \textbf{Rationale}: Ensures accountability for access attempts [https://www.researchgate.net/publication/352745109\_Audit\_Logs\_Management\_and\_Security\_-\_A\_Survey].

    \item The system shall log all access to sensitive data, and data modification such as updating, adding or deleting rule alerts, or registering new sensors or accounts. \\
    \textbf{Rationale}: Ensures sensitive data is properly handled, and traceability of critical changes.[https://www.secoda.co/glossary/audit-traceability-audit-logs].

    \item All audit logs should be retained for at least 1 years. \\
    \textbf{Rationale}: Ensures there is enough time look at incidents, or check compliance [https://auditboard.com/blog/security-log-retention-best-practices-guide].

    \item All audit logs should be protected with access controls to ensure that only authorized security can view or extract information from them. \\
    \textbf{Rationale}: Prevents critical information from being shared without authorization.[https://auditboard.com/blog/security-log-retention-best-practices-guide].
    
\end{enumerate}
% End SubSubSection

\subsubsection{Immunity Requirements}
\label{ssub:immunity_requirements}
% Begin SubSubSection
\begin{enumerate}[{SR-IM}1. ]
	\item The system shall keep receiving and storing sensor data even when the real-time processing system is down. \\
    \textbf{Rationale}: Makes sure no sensor data is lost if part of the system stops.    [https://www.cyber.gc.ca/en/cyber-security-readiness/cyber-security-readiness-goals-securing-our-most-critical-systems].

    \item The system shall immediately try again when sending sensor data fails.\\
    \textbf{Rationale}: Ensures that sensor data eventually reaches the system, even in the case of network problems.    [https://www.cyber.gc.ca/en/cyber-security-readiness/cyber-security-readiness-goals-securing-our-most-critical-systems].


\end{enumerate}
% End SubSubSection

% End SubSection

\subsection{Cultural and Political Requirements}
\label{sub:cultural_and_political_requirements}
% Begin SubSection

\subsubsection{Cultural Requirements}
\label{ssub:cultural_requirements}
% Begin SubSubSection
\begin{enumerate}[{CP-C}1. ]
	\item The system shall not allow users to create an account with an inappropriate or discriminatory name. \\
    \textbf{Rationale}:
    
    \item The system shall not use any symbols, colours, and icons in the dashboards that can be considered offensive based one ones religion, ethnicity, disability or sexual orientation.  \\
    \textbf{Rationale}:

    \item The system shall not use any symbols, colours, and icons in the dashboards that can be considered offensive based one ones religion, ethnicity, disability or sexual orientation.  \\
    \textbf{Rationale}:
\end{enumerate}
% End SubSubSection

\subsubsection{Political Requirements}
\label{ssub:political_requirements}
% Begin SubSubSection
\begin{enumerate}[{CP-P}1. ]
	\item The system shall comply with all federal, provincial, and municipal regulations regarding environmental monitoring and data sharing [https://www.canada.ca/en/environment-climate-change/services/canadian-environmental-protection-act-registry/monitoring-reporting-research/monitoring.html]. \\
    \textbf{Rationale}: Ensures legal compliance and helps to avoid political conflicts.
    	
    \item The system shall be transparent in how it collects and stores data by giving government authorities access to such data and system reports [https://www.tbs-sct.canada.ca/pol/doc-eng.aspx?id=28108]. \\
    \textbf{Rationale}: Enables accountability to political authorities. 
        	
    \item The system shall provide ways to share environmental data to other authorized groups. \\
    \textbf{Rationale}: Supports collaboration between government organizations.
\end{enumerate}
% End SubSubSection

\subsection{Legal Requirements}
\label{sub:legal_requirements}
% Begin SubSection

\subsubsection{Compliance Requirements}
\label{ssub:compliance_requirements}
% Begin SubSubSection
\begin{enumerate}[{LR-COMP}1. ]
	\item 
\end{enumerate}
% End SubSubSection

\subsubsection{Standards Requirements}
\label{ssub:standards_requirements}
% Begin SubSubSection
\begin{enumerate}[{LR-STD}1. ]
	\item 
\end{enumerate}
% End SubSubSection

% End SubSection

% End Section
\section{Innovative Feature}
\label{sec:innovative_feature}
An innovative feature that could be added is an included simulation feature that would allow city operators to adjust environmental parameters and observe the projected outcome in a visual dashboard overlaid on top of the city. This would help city operators plan response strategies, visualize worst case outcomes, and help train employees without impacting live data.

\appendix
\section{Division of Labour}
\label{sec:division_of_labour}
% Begin Section
Include a Division of Labour sheet which indicates the contributions of each team member. This sheet must be signed by all team members. \\

\noindent \textbf{Benjamin Bloomfield} 
\begin{itemize}
    \item Main point
    \begin{itemize}
        \item Sub point
    \end{itemize}
\end{itemize}

\noindent \textbf{Vikram Chandar}
\begin{itemize}
    \item 
\end{itemize}

\noindent \textbf{Trisha Panchal}
\begin{itemize}
    \item 
\end{itemize}

\noindent \textbf{Yug Vashisth}
\begin{itemize}
    \item 
\end{itemize}

\noindent \textbf{Hamza Nadeem}
\begin{itemize}
    \item 
\end{itemize}
% End Section

%\newpage
%\section*{IMPORTANT NOTES}
%\begin{itemize}
%	\item Be sure to include all sections of the template in your document regardless whether you have something to write for each or not
%	\begin{itemize}
%		\item If you do not have anything to write in a section, indicate this by the \emph{N/A}, \emph{void}, \emph{none}, etc.
%	\end{itemize}
%	\item Uniquely number each of your requirements for easy identification and cross-referencing
%	\item Highlight terms that are defined in Section~1.3 (\textbf{Definitions, Acronyms, and Abbreviations}) with \textbf{bold}, \emph{italic} or \underline{underline}
%	\item For Deliverable 1, please highlight, in some fashion, all (you may have more than one) creative and innovative features. Your creative and innovative features will generally be described in Section~2.2 (\textbf{Product Functions}), but it will depend on the type of creative or innovative features you are including.
%\end{itemize}


\end{document}
%------------------------------------------------------------------------------