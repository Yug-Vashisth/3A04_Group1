\documentclass[]{article}

% Imported Packages
%------------------------------------------------------------------------------
\usepackage{amssymb}
\usepackage{amstext}
\usepackage{amsthm}
\usepackage{amsmath}
\usepackage{enumerate}
\usepackage{fancyhdr}
\usepackage[margin=1in]{geometry}
\usepackage{graphicx}
%\usepackage{extarrows}
%\usepackage{setspace}
%\usepackage{xcolor}
\usepackage{color}
%------------------------------------------------------------------------------

% Header and Footer
%------------------------------------------------------------------------------
\pagestyle{plain}  
\renewcommand\headrulewidth{0.4pt}                                      
\renewcommand\footrulewidth{0.4pt}                                    
%------------------------------------------------------------------------------

% Title Details
%------------------------------------------------------------------------------
\title{Deliverable \#1 : Software Requirement Specification (SRS)}
\author{SE 3A04: Software Design III -- Large System Design}
\date{}
                            
%------------------------------------------------------------------------------

% Document
%------------------------------------------------------------------------------
\begin{document}

\maketitle	
\noindent{\bf Tutorial Number:} T01\\
{\bf Group Number:} G01 \\
{\bf Group Members:} 
\begin{itemize}
	\item Benjamin Bloomfield
	\item Vikram Chandar
    \item Trisha Panchal
    \item Yug Vashisth
    \item Hamza Nadeem
\end{itemize}

\section*{IMPORTANT NOTES}
\begin{itemize}
	\item Be sure to include all sections of the template in your document regardless whether you have something to write for each or not
	\begin{itemize}
		\item If you do not have anything to write in a section, indicate this by the \emph{N/A}, \emph{void}, \emph{none}, etc.
	\end{itemize}
	\item Uniquely number each of your requirements for easy identification and cross-referencing
	\item Highlight terms that are defined in Section~1.3 (\textbf{Definitions, Acronyms, and Abbreviations}) with \textbf{bold}, \emph{italic} or \underline{underline}
	\item For Deliverable 1, please highlight, in some fashion, all (you may have more than one) creative and innovative features. Your creative and innovative features will generally be described in Section~2.2 (\textbf{Product Functions}), but it will depend on the type of creative or innovative features you are including.
\end{itemize}

\newpage
\section{Introduction}
\label{sec:introduction}
% Begin Section

\begin{itemize}
	\item Provide an overview of the document/SRS.
\end{itemize}


\subsection{Purpose}
\label{sub:purpose}
% Begin SubSection
\begin{itemize}
	\item Specify the purpose of the SRS.
	\item Specify the intended audience for the SRS.
\end{itemize}
% End SubSection

\subsection{Scope}
\label{sub:scope}
% Begin SubSection
\begin{itemize}
	\item Identify the software product(s) to be produced, and name each (e.g., Host DBMS, Report Generator, etc.)
	\item Explain what the software product(s) will do (and, if necessary, also state what they will not do).
	\item Describe the application of the software being specified, including relevant benefits, objectives, and goals.
%	\item Be consistent with similar statements in higher-level specifications (e.g., the system requirements specification), if they exist
\end{itemize}
% End SubSection

\subsection{Definitions, Acronyms, and Abbreviations}
\label{sub:definitions_acronyms_and_abbreviations}
% Begin SubSection
\begin{itemize}
	\item Provide the definitions of all terms, acronyms, and abbreviations required to properly interpret the SRS.
	\item This should be in alphabetical order.
\end{itemize}
% End SubSection

\subsection{References}
\label{sub:references}
% Begin SubSection
\begin{itemize}
	\item Provide a complete list of all documents referenced elsewhere in the SRS.
	\item Identify each document by title, report number (if applicable), date, and publishing organization.
	\item Specify the sources from which the references can be obtained.
	\item Order this list in some sensible manner (alphabetical by author, or something else that makes more sense).
\end{itemize}
% End SubSection

\subsection{Overview}
\label{sub:overview}
% Begin SubSection
\begin{itemize}
	\item Describe what the remainder of the document/SRS contains.\\
	(e.g. "Section 2 discusses...Section 3...")
%	\item Explain how the SRS is organized
\end{itemize}
% End SubSection

% End Section

\section{Overall Product Description}
\label{sec:overall_description}
% Begin Section

\begin{itemize}
	\item This section should describe the general factors that affect the product and its requirements. 
	\item It does not state specific requirements.
	\item It provides a \emph{background} for those requirements and makes them easier to understand.
\end{itemize}


\subsection{Product Perspective}
\label{sub:product_perspective}
% Begin SubSection
\begin{itemize}
	\item Put the product into perspective with other related products, i.e., context
	\item If the product is independent and totally self-contained, it should be stated here
	\item If the SRS defines a product that is a component of a larger system, then this subsection should relate the requirements of that larger system to the functionality of the software being developed. Identify interfaces between that larger system and the software to be developed.
	\item A block diagram showing the major components of the larger system, interconnections, and external interfaces can be helpful
\end{itemize}
% End SubSection

\subsection{Product Functions}
\label{sub:product_functions}
% Begin SubSection
\begin{itemize}
	\item Provide a \emph{summary} of the major functions that the software will perform.
	\begin{itemize}
		\item \textbf{Example}: An SRS for an accounting program may use this part to address customer account maintenance, customer statement, and invoice preparation without mentioning the vast amount of detail that each of those functions requires.
	\end{itemize}
	\item Functions should be organized in a way that makes the list of functions understandable to the customer or to anyone else reading the document for the first time 
	\item Present the functions in a list format - each item should be one function, with a brief description of it
	\item Textual or graphical methods can be used to show the different functions and their relationships
	\begin{itemize}
		\item Such a diagram is not intended to show a design of a product, but simply shows the logical relationships among variables
	\end{itemize} 
\end{itemize}
% End SubSection

\subsection{User Characteristics}
\label{sub:user_characteristics}

\begin{itemize}
	\item Describe those general characteristics of the intended users of the product including educational level, experience, and technical expertise 
	\item Since there will be many users, you may wish to divide into different user types or personas
	\item Do not state specific requirements, but rather provide the reasons why certain specific requirements are later specified
\end{itemize}

\subsubsection{User I: City Operators}
City operators analyze the web dashboard to monitor key environmental conditions, and respond accordingly to alerts generated by the system.
\begin{itemize}
    \item \textbf{Education Level}: City operators are expected to have some level of post-secondary education, in environmental science, engineering, urban planning, or another related field of study.
    \item \textbf{Experience}: Prior experience  working in a similar role, such as environmental monitoring or public safety operations. Such users should be familiar with interpreting geographical maps, environmental metrics, charts, and trends to identify irregular environmental conditions.
    \item \textbf{Technical Expertise}: City operators are expected to have intermediate computer proficiency, such as the ability to navigate web-based applications, interact with dashboards, and operate data visualization tools.
\end{itemize}

\subsubsection{User II: Public Users}
The general public and citizens can access non-sensitive environmental information through public interfaces that use the read-only REST API, for awareness and safety purposes.
\begin{itemize}
    \item \textbf{Education Level}: Public users are assumed to have basic literacy skills. Basic foundations in listening, reading, writing, and speaking are all the user needs to use the platform without any challenges.
    \item \textbf{Experience}: The user is  expected to have little to no prior experience using monitoring systems. The system is designed to be user-friendly to ensure that interactions are straightforward for any user.
    \item \textbf{Technical Expertise}: Basic technical skills, such as understanding how to use a web browser, navigate a webpage, and view digital displays is all that is necessary.
\end{itemize}

\subsubsection{User III: Administrators}
System administrators are responsible for building and maintaining the system. Their role focuses specifically on ensuring the system operates securely, behaves correctly, and is maintainable over time.
\begin{itemize}
    \item \textbf{Education Level}: System administrators should have at least a post-secondary education in software engineering, computer science, or another similar technical field.
    \item \textbf{Experience}: They are expected to have prior experience maintaining or managing software systems in a professional or academic setting. This experience includes responsibilities related to access control, system configuration and operational oversight. Such users also should understand how to define system policies and supervise automated processes like logging and alert handling. 
    \item \textbf{Technical Expertise}: Advanced technical skills are required for system administrators. They should be familiar with configurable rule-based systems, authentication methods, role-based access control (RBAC) models, and audit logging.
\end{itemize}

\subsubsection{User IV: Third-Party Developer}
Third-party developers request and interpret environmental data from the public API to help build external applications. 
\begin{itemize}
    \item \textbf{Education Level}: Developers should have a post-secondary education in software engineering, computer science, or equivalent professional experience.
    \item \textbf{Experience}: The third-party developers are expected to have experience developing software that uses REST APIs and works with JSON data
    \item \textbf{Technical Expertise}: Proficient technical skills in web application development and API integration, as well as the ability to comprehend API documentation.
\end{itemize}
% End SubSection

\subsection{Constraints}
\label{sub:constraints}
% Begin SubSection
\begin{itemize}
	\item Provide a general description of any constraints that will limit the developer's options
\end{itemize}

\begin{enumerate}
    \item \textbf{Budget}: The project is constrained by a budget of \$0. This impacts what features the developer can implement, as all technologies, frameworks, cloud services, and other development tools must be free for the team to use.
    \item \textbf{Project Time}: The development of the system is constrained to the Winter 2026 academic calendar. This limits the overall scope of the system, and the amount of features that can be implemented at launch. It shapes the overall development timeline, and the ability to test and optimize. 
    \item \textbf{Communication Protocol}: All sensor telemetry must use the MQTT protocol. This restricts the system’s communication model and requires that incoming messages align with the predefined data formats that MQTT messaging supports.
    \item \textbf{Personal Identifiable Information}: The system is constrained to operate under privacy-by-design principles, and must not collect, process, or store any personally identifiable information. Environmental data must be aggregated and associated with general city zones rather than precise locations.
    \item \textbf{Public API Scope}: The public and third-party facing API is constrained to be read-only, meaning that they can not submit data or modify the state. It must only provides access to aggregated, non-sensitive environmental data.
    \item \textbf{Simulated Hardware}: Physical IoT sensor hardware is not available for this project and must be be generated using software simulations. This influences design decisions as the functionality cannot be verified using real-world sensors.
    \item \textbf{Deployment Environment}: The system is constrained to a cloud architecture using available platforms and tools suitable for educational use. Therefore, developers cannot deploy the system on local machines or private servers.
\end{enumerate}
% End SubSection

\subsection{Assumptions and Dependencies}
\label{sub:assumptions_and_dependencies}
% Begin SubSection
\begin{itemize}
	\item List any assumptions you made in interpreting what the software being developed is aiming to achieve
	\item List any other assumptions you made that, if it fails to hold, could require you to change the requirements
	%\item List each of the factors that affect the requirements stated in the SRS
	%\item These factors are not design constraints on the software but are, rather, any changes to them that can affect the requirements in the SRS
	\begin{itemize}
		\item \textbf{Example}: An assumption may be that a specific operating system will be available on the hardware designated for the software product. If, in fact, the operating system is not available, the SRS would then have to change accordingly.
	\end{itemize}
\end{itemize}
% End SubSection

\subsection{Apportioning of Requirements}
\label{sub:apportioning_of_requirements}
% Begin SubSection
\begin{itemize}
	\item Identify requirements that may be delayed until future versions of the system
\end{itemize}
% End SubSection

% End Section
\section{Use Case Diagram}
\label{sec:use_case_diagram}
% Begin Section
\begin{itemize}
	\item Provide the use case diagram for the system being developed.
	\item You do not need to provide the textual description of any of the use cases here (these will be specified under "Highlights of Functional Requirements").
%	\item Provide \emph{one} use case diagram for the most important Business Event.
%	\item The text of all use cases will be specified under "Highlights of Functional Requirements"
\end{itemize}
%In this section, select the most important Business Event that your system responds to and give its use case diagram.  Only one use case diagram is needed.  Give a brief textual description of the use case without repeating what is in the scenarios of the corresponding Business Event.

%
%
%
%This section should provide a use case diagram for your application. 
%\begin{enumerate}[a)]
%	\item Each use case appearing in the diagram should be accompanied by a text description. 
%\end{enumerate}
%% End Section

\section{Highlights of Functional Requirements}
\label{sec:functional_requirements}
% Begin Section
\begin{itemize}
	\item Specify all use cases (or other scenarios triggered by other events), organized by Business Event. 
	\item For each Business Event, show the scenario from every Viewpoint. You should have the same set of Viewpoints across all Business Events. If a Viewpoint doesn't participate, write N/A so we know you considered it still. You can choose how to present this - keep in mind it should be easy to follow. 
	\item At the end, combine them all into a Global Scenario.
	%\item Specify the "use cases" (or other triggering events) organized by Business Event. (The Global Scenario is what you might think of as a use case). Be sure to consider Business Events that aren't just triggered by users with goals (e.g. something happens in the environment that your system needs to respond to)
	\item Your focus should be on what the system needs to do, not how to do it. Specify it in enough detail that it clearly specifies what needs to be accomplished, but not so detailed that you start programming or making design decisions.
	\item Keep the length of each use case (Global Scenario) manageable. If it's getting too long, split into sub-cases.
	\item You are \emph{not} specifying a complete and consistent set of functional requirements here. (i.e. you are providing them in the form of use cases/global scenarios, not a refined list). For the purpose of this project, you do not need to reduce them to a list; the global scenarios format is all you need.
	\item Red text below is just to highlight where you need to insert a scenario - don't actually write it all in red.
\end{itemize}

\noindent {\bf Main Business Events:} List out all the main business events you are presenting. If you sub-divided into smaller ones, you don't need to include the smaller ones in this list.\\

\noindent {\bf Viewpoints:} List out all the viewpoints you will be considering.\\

\noindent {\bf Interpretation:} Specify any liberties you took in interpreting business events, if necessary.\\

\begin{enumerate}[{\bf BE1.}]
	\item Define a New Alert Rule. \#1 \\         
    \textbf{Pre-condition}: The administrator has an authenticated account and there is no alert rule that already exists for the same threshold.
    
		\begin{enumerate}[{\bf VP1.}]
			\item City Operators \#1 
                \begin{quote}
                    N/A
                \end{quote}
                
			\item Public Users \#2 
				\begin{quote}
				    N/A
				\end{quote}
                
			\item Administrators \#3 \\
				\textbf{Main Success Scenario} 
                    \begin{quote}
                    \begin{enumerate}[1.]
                        \item Administrator opens the dashboard.
                        \item System prompts the administrator to log in.
                        \item Administrator enters login information.
                        \item System authenticates the administrator.
                        \item System displays the dashboard.
                        \item Administrator selects \textit{Manage Alerts} tab
                        \item System displays the existing alert rules, and the option to create a new rule.
                        \item Administrator selects \textit{Create New Alert Rule}.
                        \item System displays the alert rule setup form.
                        \item Administrator enters the new alert rule information.
                        \item Administrator submits the alert rule to be added to the system.
                        \item System validates the form.
                        \item System stores the new alert rule.
                        \item System reports to administrator that the alert has been successfully created.
                    \end{enumerate}
                    \end{quote}
                \textbf{Secondary Scenario} 
                \begin{quote}
                    \begin{description}
                        \item[\textmd{4i.}] System fails to authenticate the administrator.
                        \begin{description}
                            \item[\textmd{4i.1}] System rejects the authentication attempt.
                            \item[\textmd{4i.2}] System displays an authentication failure message.
                            \item[\textmd{4i.3}] System prompts the administrator to re-enter credentials.
                            \item[\textmd{4i.4}] Administrator selects try again.
                            \item[\textmd{4i.5}] System tracks the number of consecutive failed sign-in attempts.
                            \item[\textmd{4i.6}] Return to BE1-3.
                        \end{description}
                        \item[\textmd{4ii.}] Maximum amount of sign-in attempts have been reached. 
                        \begin{description}
                            \item[\textmd{4ii.1}] System temporarily locks the account.
                            \item[\textmd{4ii.2}] System displays a message indicating that access is restricted.
                            \item[\textmd{4ii.3}] System provides instructions for contacting IT Support to resolve the issue.
                        \end{description}
                        \item[\textmd{12i.}] System detects invalid or incomplete data.
                        \begin{description}
                            \item[\textmd{12i.1}] System prevents the alert rule from being submitted.
                            \item[\textmd{12i.2}] System displays an error message describing the issue.
                            \item[\textmd{12i.3}] System highlights the section(s) that need to be changed.
                            \item[\textmd{12i.4}] Return to BE1-10.
                        \end{description}
                        \item[\textmd{13i.}] System fails to store the added alert rule.
                        \begin{description}
                            \item[\textmd{13i.1}] System displays an error message that the alert rule could not be created.
                            \item[\textmd{13i.2}] System provides instructions for contacting IT Support.
                        \end{description}
                    \end{description}
                \end{quote}
                
			\item Third-Party Developers \#4 
				\begin{quote}
				    N/A
				\end{quote}
                
			\item Environmental Monitoring Department \#5 
				\begin{quote}
				    N/A
				\end{quote}

			\item IT Support \#6 
                \item[\textmd{4ii}] System should allow IT Support to restore administrator access after verifying their identity.
                \item[\textmd{13i}] System should provide IT Support with error logs to resolve the alert rule storage failures.
		\end{enumerate}
        
        {\bf Global Scenario:}
        	\begin{quote}
        	    \textbf{Pre-condition}: Define a New Alert Rule.. \\
                \textbf{Main Success Scenario} 
                    \begin{quote}
                    \begin{enumerate}[1.]
                        \item Administrator opens the dashboard.
                        \item System prompts the administrator to log in.
                        \item Administrator enters login information.
                        \item System authenticates the administrator.
                        \item System displays the dashboard.
                        \item Administrator selects \textit{Manage Alerts} tab
                        \item System displays the existing alert rules, and the option to create a new rule.
                        \item Administrator selects \textit{Create New Alert Rule}.
                        \item System displays the alert rule setup form.
                        \item Administrator enters the new alert rule information.
                        \item Administrator submits the alert rule to be added to the system.
                        \item System validates the form.
                        \item System stores the new alert rule.
                        \item System reports to administrator that the alert has been successfully created.
                    \end{enumerate}
                    \end{quote}
                \textbf{Secondary Scenario} 
                \begin{quote}
                    \begin{description}
                        \item[\textmd{4i.}] System fails to authenticate the administrator.
                        \begin{description}
                            \item[\textmd{4i.1}] System rejects the authentication attempt.
                            \item[\textmd{4i.2}] System displays an authentication failure message.
                            \item[\textmd{4i.3}] System prompts the administrator to re-enter credentials.
                            \item[\textmd{4i.4}] Administrator selects try again.
                            \item[\textmd{4i.5}] System tracks the number of consecutive failed sign-in attempts.
                            \item[\textmd{4i.6}] Return to BE1-3.
                        \end{description}
                        \item[\textmd{4ii.}] Maximum amount of sign-in attempts have been reached. 
                        \begin{description}
                            \item[\textmd{4ii.1}] System temporarily locks the account.
                            \item[\textmd{4ii.2}] System displays a message indicating that access is restricted.
                            \item[\textmd{4ii.3}] System provides instructions for contacting IT Support.
                        \end{description}
                        \item[\textmd{12i.}] System detects invalid or incomplete data.
                        \begin{description}
                            \item[\textmd{12i.1}] System prevents the alert rule from being submitted.
                            \item[\textmd{12i.2}] System displays an error message describing the issue.
                            \item[\textmd{12i.3}] System highlights the section(s) that need to be changed.
                            \item[\textmd{12i.4}] Return to BE1-10.
                        \end{description}
                        \item[\textmd{13i.}] System fails to store the added alert rule.
                        \begin{description}
                            \item[\textmd{13i.1}] System displays an error message that the alert rule could not be created.
                            \item[\textmd{13i.2}] System provides instructions for contacting IT Support.
                        \end{description}
                    \end{description}
                \end{quote}
        	\end{quote}
\end{enumerate}

\begin{enumerate}[{\bf BE2.}]
	\item Update an Existing Alert Rule. \#2 \\         
    \textbf{Pre-condition}: There must exist at least one alert rule in the system. The administrator is authenticated and logged into the system.
    
		\begin{enumerate}[{\bf VP1.}]
			\item City Operators \#1 
                \begin{quote}
                    N/A
                \end{quote}
                
			\item Public Users \#2 
				\begin{quote}
				    N/A
				\end{quote}
                
			\item Administrators \#3 \\
				\textbf{Main Success Scenario} 
                    \begin{quote}
                    \begin{enumerate}[1.]
                        \item System displays the dashboard.
                        \item Administrator selects \textit{Manage Alerts} tab
                        \item System displays the existing alert rules.
                        \item Administrator selects an alert rule to update.
                        \item Administrator modifies the alert rule parameters.
                        \item Administrator submits the updated alert rule.
                        \item System validates the updated alert rule.
                        \item System updates the alert rule in the systems database.
                        \item System reports to the administrator that the alert rule has been successfully updated.
                    \end{enumerate}
                    \end{quote}
                \textbf{Secondary Scenario} 
                \begin{quote}
                    \begin{description}
                        \item[\textmd{7i.}] Updated alert information is invalid or incomplete.
                        \begin{description}
                            \item[\textmd{7i.1}] System prevents the updated alert rule from being submitted.
                            \item[\textmd{7i.2}] System displays an error message describing the issue.
                            \item[\textmd{7i.3}] System highlights the section(s) that need to be changed.
                            \item[\textmd{7i.4}] Return to BE2-5.
                        \end{description}
                        \item[\textmd{8i.}] System fails to update the alert rule.
                        \begin{description}
                            \item[\textmd{8i.1}] System displays an error message indicating that the alert rule update failed.
                            \item[\textmd{8i.2}] System provides instructions for contacting IT Support.
                        \end{description}
                    \end{description}
                \end{quote}
                
			\item Third-Party Developers \#4
				\begin{quote}
				    N/A
				\end{quote}
                
			\item Environmental Monitoring Department \#5
				\begin{quote}
				    N/A
				\end{quote}

			\item IT Support \#6
            	\begin{quote}
				    8i. System provides IT Support with error logs to resolve the alert rule update failures.
				\end{quote}

		\end{enumerate}
        
        {\bf Global Scenario:}
        	\begin{quote}
        	    \textbf{Pre-condition}: There must exist at least one alert rule in the system. The administrator is authenticated and logged into the system. \\
                \textbf{Main Success Scenario} 
                    \begin{quote}
                    \begin{enumerate}[1.]
                        \item System displays the dashboard.
                        \item Administrator selects \textit{Manage Alerts} tab
                        \item System displays the existing alert rules.
                        \item Administrator selects an alert rule to update.
                        \item Administrator modifies the alert rule parameters.
                        \item Administrator submits the updated alert rule.
                        \item System validates the updated alert rule.
                        \item System updates the alert rule in the systems database.
                        \item System reports to the administrator that the alert rule has been successfully updated
                    \end{enumerate}
                    \end{quote}
                \textbf{Secondary Scenario} 
                \begin{quote}
                    \begin{description}
                        \item[\textmd{7i.}] Updated alert information is invalid or incomplete.
                        \begin{description}
                            \item[\textmd{7i.1}] System prevents the updated alert rule from being submitted.
                            \item[\textmd{7i.2}] System displays an error message describing the issue.
                            \item[\textmd{7i.3}] System highlights the section(s) that need to be changed.
                            \item[\textmd{7i.4}] Return to BE2-5.
                        \end{description}
                        \item[\textmd{8i.}] System fails to update the alert rule.
                        \begin{description}
                            \item[\textmd{8i.1}] System displays an error message that that the alert rule update failed.
                            \item[\textmd{8i.2}] System provides instructions for contacting IT Support.
                        \end{description}
                    \end{description}
                \end{quote}
        	\end{quote}
\end{enumerate}

\begin{enumerate}[{\bf BE3.}]
	\item Remove or Disable an Alert Rule \#3 \\         
    \textbf{Pre-condition}: There must exist at least one alert rule in the system. The administrator is authenticated and logged into the system.
    
		\begin{enumerate}[{\bf VP1.}]
			\item City Operators \#1 
                \begin{quote}
                    N/A
                \end{quote}
                
			\item Public Users \#2 
				\begin{quote}
				    N/A
				\end{quote}
                
			\item Administrators \#3 \\
				\textbf{Main Success Scenario} 
                    \begin{quote}
                    \begin{enumerate}[1.]
                        \item System displays the dashboard.
                        \item Administrator selects \textit{Manage Alerts} tab
                        \item System displays the existing alert rules.
                        \item Administrator selects an alert rule to remove or disable.
                        \item System prompts the administrator for confirmation.
                        \item Administrator confirms the action.
                        \item System removes or disables the alert rule.
                        \item System reports that the alert has been successfully removed or disabled.
                    \end{enumerate}
                    \end{quote}
                \textbf{Secondary Scenario} 
                \begin{quote}
                    \begin{description}
                        \item[\textmd{6i.}] Administrator cancels the confirmation.
                        \begin{description}
                            \item[\textmd{6i.1}] System does not remove or disable the alert rule.
                            \item[\textmd{6i.2}] System returns to the list of existing alert rules.
                        \end{description}
                        \item[\textmd{7i.}] System fails to remove or disable the alert rule.
                        \begin{description}
                            \item[\textmd{7i.1}] System displays an error message that the action failed.
                            \item[\textmd{7i.2}] System provides instructions for contacting IT Support.
                        \end{description}
                    \end{description}
                \end{quote}
                
			\item Third-Party Developers \#4
				\begin{quote}
				    N/A
				\end{quote}
                
			\item Environmental Monitoring Department \#5
				\begin{quote}
				    N/A
				\end{quote}

			\item IT Support \#6
                \item[\textmd{7i}] System provides IT Support with system logs to investigate and resolve alert rule removal or disable failures.
		\end{enumerate}
        
        {\bf Global Scenario:}
        	\begin{quote}
        	    \textbf{Pre-condition}: There must exist at least one alert rule in the system. The administrator is authenticated and logged into the system. \\
                \textbf{Main Success Scenario} 
                    \begin{quote}
                    \begin{enumerate}[1.]
                        \item System displays the dashboard.
                        \item Administrator selects \textit{Manage Alerts} tab
                        \item System displays the existing alert rules.
                        \item Administrator selects an alert rule to remove or disable.
                        \item System prompts user for confirmation.
                        \item Administrator confirms the action.
                        \item System removes or disables the alert rule.
                        \item System reports that the alert has been successfully removed or disabled.
                    \end{enumerate}
                    \end{quote}
                \textbf{Secondary Scenario} 
                \begin{quote}
                    \begin{description}
                        \item[\textmd{6i.}] Administrator cancels the confirmation.
                        \begin{description}
                            \item[\textmd{6i.1}] System does not remove or disable the alert rule.
                            \item[\textmd{6i.2}] System returns to the list of existing alert rules.
                        \end{description}
                        \item[\textmd{7i.}] System fails to remove or disable the alert rule.
                        \begin{description}
                            \item[\textmd{7i.1}] System displays an error message that the action failed.
                            \item[\textmd{7i.2}] System provides instructions for contacting IT Support.
                        \end{description}
                    \end{description}
                \end{quote}
        	\end{quote}
\end{enumerate}

\begin{enumerate}[{\bf BE4.}]
	\item City Operator Acknowledges an Alert \#4 \\      
    \textbf{Pre-condition}: An alert has been generated by the system and is marked as \textit{Active}. The city operator is authenticated and logged into the system.
    
		\begin{enumerate}[{\bf VP1.}]
			\item City Operators \#1 
                \begin{quote}
                    N/A
                \end{quote}
                
			\item Public Users \#2 
				\begin{quote}
				    N/A
				\end{quote}
                
			\item Administrators \#3 \\
				\textbf{Main Success Scenario} 
                    \begin{quote}
                    \begin{enumerate}[1.]
                        \item System displays the dashboard showing the active alerts.
                        \item City operator selects an active alert from the dashboard.
                        \item System displays the alert details.
                        \item City operator selects the option to acknowledge the alert.
                        \item System updates the alert status to \textit{Acknowledged}.
                        \item System confirms that the alert has been acknowledged. 
                    \end{enumerate}
                    \end{quote}
                \textbf{Secondary Scenario} 
                \begin{quote}
                    \begin{description}
                        \item[\textmd{4i.}] Alert has already been acknowledged by another operator.
                        \begin{description}
                            \item[\textmd{4i.1}] System refreshes the alert status.
                            \item[\textmd{4i.2}] System informs the city operator that the alert is already acknowledged.
                        \end{description}
                        \item[\textmd{5i.}] System fails to update the alert status.
                        \begin{description}
                            \item[\textmd{5i.1}] System displays an error message to city operator.
                            \item[\textmd{5i.2}] Alert remains in \textit{Active} status.
                            \item[\textmd{5i.3}] System provides instructions for contacting IT Support.
                        \end{description}
                    \end{description}
                \end{quote}
                
			\item Third-Party Developers \#4
				\begin{quote}
				    N/A
				\end{quote}
                
			\item Environmental Monitoring Department \#5
				\begin{quote}
				    N/A
				\end{quote}

			\item IT Support \#6
                \item[\textmd{5i}] System provides IT Support with system logs to investigate and resolve alert acknowledgment issues.
		\end{enumerate}
        
        {\bf Global Scenario:}
        	\begin{quote}
        	    \textbf{Pre-condition}: An alert has been generated by the system and is marked as \textit{Active}. The city operator is authenticated and logged into the system. \\
                \textbf{Main Success Scenario} 
                    \begin{quote}
                    \begin{enumerate}[1.]
                        \item System displays the dashboard showing the active alerts.
                        \item City operator selects an active alert from the dashboard.
                        \item System displays the alert details.
                        \item City operator selects the option to acknowledge the alert.
                        \item System updates the alert status to \textit{Acknowledged}.
                        \item System confirms that the alert has been acknowledged. 
                    \end{enumerate}
                    \end{quote}
                \textbf{Secondary Scenario} 
                \begin{quote}
                    \begin{description}
                        \item[\textmd{4i.}] Alert got acknowledged by another operator.
                        \begin{description}
                            \item[\textmd{4i.1}] System refreshes the alert status.
                            \item[\textmd{4i.2}] System informs the city operator that the alert is already acknowledged.
                        \end{description}
                        \item[\textmd{5i.}] System fails to update the alert status.
                        \begin{description}
                            \item[\textmd{5i.1}] System displays an error message to city operator.
                            \item[\textmd{5i.2}] Alert remains in \textit{Active} status.
                            \item[\textmd{5i.3}] System provides instructions for contacting IT Support.
                        \end{description}
                    \end{description}
                \end{quote}
        	\end{quote}
\end{enumerate}

\begin{enumerate}[{\bf BE5.}]
	\item City Operator Resolves an Alert \#5 \\      
    \textbf{Pre-condition}: An alert has been generated by the system and has been acknowledged by a city operator. The city operator is authenticated and logged into the system.
    
		\begin{enumerate}[{\bf VP1.}]
			\item City Operators \#1 
                \begin{quote}
                    N/A
                \end{quote}
                
			\item Public Users \#2 
				\begin{quote}
				    N/A
				\end{quote}
                
			\item Administrators \#3 \\
				\textbf{Main Success Scenario} 
                    \begin{quote}
                    \begin{enumerate}[1.]
                        \item System displays the dashboard showing the active alerts.
                        \item City operator selects an alert.
                        \item System displays the alert details.
                        \item City operator marks the alert as resolved.
                        \item System updates the alert status to \textit{Resolved}.
                        \item System records the time and the operators details.
                        \item System moves the alert to the \textit{Completed} page.
                        \item System confirms that the alert has been resolved.
                    \end{enumerate}
                    \end{quote}
                \textbf{Secondary Scenario} 
                \begin{quote}
                    \begin{description}
                        \item[\textmd{4i.}] Alert got resolved by another operator
                        \begin{description}
                            \item[\textmd{4i.1}] System refreshes the alert status.
                            \item[\textmd{4i.2}] System informs the city operator that the alert is already acknowledged.
                        \end{description}
                        \item[\textmd{5i.}] System fails to update the alert status.
                        \begin{description}
                            \item[\textmd{5i.1}] System displays an error message to city operator.
                            \item[\textmd{5i.2}] Alert remains in \textit{Acknowledged} status.
                            \item[\textmd{5i.2}] System provides instructions for contacting IT Support.
                        \end{description}
                    \end{description}
                \end{quote}
                
			\item Third-Party Developers \#4
				\begin{quote}
				    N/A
				\end{quote}
                
			\item Environmental Monitoring Department \#5
				\begin{quote}
				    N/A
				\end{quote}

			\item IT Support \#6
                \item[\textmd{5i}] System provides IT Support with system logs to investigate and resolve alert resolution failures or audit logging issues.
		\end{enumerate}
        
        {\bf Global Scenario:}
        	\begin{quote}
        	    \textbf{Pre-condition}: An alert has been generated by the system and has been acknowledged by a city operator. The city operator is authenticated and logged into the system. \\
                \textbf{Main Success Scenario} 
                    \begin{quote}
                    \begin{enumerate}[1.]
                        \item System displays the dashboard showing the active alerts.
                        \item City operator selects an alert.
                        \item System displays the alert details.
                        \item City operator marks the alert as resolved.
                        \item System updates the alert status to \textit{Resolved}.
                        \item System records the time and the operators details.
                        \item System moves the alert to the \textit{Completed} page.
                        \item System confirms that the alert has been resolved.
                    \end{enumerate}
                    \end{quote}
                \textbf{Secondary Scenario} 
                \begin{quote}
                    \begin{description}
                        \item[\textmd{4i.}] Alert got resolved by another operator
                        \begin{description}
                            \item[\textmd{4i.1}] System refreshes the alert status.
                            \item[\textmd{4i.2}] System informs the city operator that the alert is already acknowledged.
                        \end{description}
                        \item[\textmd{5i.}] System fails to update the alert status.
                        \begin{description}
                            \item[\textmd{5i.1}] System displays an error message to city operator.
                            \item[\textmd{5i.2}] Alert remains in \textit{Acknowledged} status.
                            \item[\textmd{5i.2}] System provides instructions for contacting IT Support.
                        \end{description}
                    \end{description}
                \end{quote}
        	\end{quote}
\end{enumerate}

%	Below, we organize by Business Event.
%	\begin{enumerate}[{BE}1.]
%		\item Business Event name
%		\begin{enumerate}[{VP1}.1]
%			\item Viewpoint name \newline
%			\noindent\fbox{%
%				\parbox{0.5\textwidth}{%
%					\begin{itemize}
%						\item {\bf $S_{1}$:} Initial response of the system to the Business Event
%						\item {\bf $E_{1}$:}  Reaction of the environment to $S_{1}$
%						\item {\bf $S_{2}$:}  Response of the system to $E_{1}$
%						\item {\bf $E_{2}$:}  Reaction of the environment to $S_{2}$
%						\item[] $\cdots$
%						\item {\bf $S_{n}$:}  Response of the system to $E_{(n-1)}$
%						\item {\bf $E_{n}$:}  Reaction of the environment to $E_{(n-1)}$
%						\item {\bf $S_{(n+1)}$:} Final response of the system concluding its function regarding the Business Event
%					\end{itemize}
%				}%
%			}
%			\item Viewpoint name\newline
%			\noindent\fbox{%
%				\parbox{0.5\textwidth}{%
%					\begin{itemize}
%						\item {\bf $S_{1}$:} Initial response of the system to the Business Event
%						\item {\bf $E_{1}$:}  Reaction of the environment to $S_{1}$
%						\item {\bf $S_{2}$:}  Response of the system to $E_{1}$
%						\item {\bf $E_{2}$:}  Reaction of the environment to $S_{2}$
%						\item[] $\cdots$
%						\item {\bf $S_{k}$:}  Response of the system to $E_{(k-1)}$
%						\item {\bf $E_{k}$:}  Reaction of the environment to $E_{(k-1)}$
%						\item {\bf $S_{(k+1)}$:} Final response of the system concluding its function regarding the Business Event
%					\end{itemize}
%				}%
%			}
%			\item \dots
%			\item \dots
%			\item \dots
%			\item[\dots]
%		\end{enumerate}	
%		\item[] {\bf Global Scenario of {\it Business Event Name}:} It is the scenario corresponding to the integration of all the above scenarios from the different Viewpoints of the Business Event BE1.\newline
%		\noindent\fbox{%
%			\parbox{0.5\textwidth}{%
%				\begin{itemize}
%					\item {\bf $S_{1}$:} Initial response of the system to the Business Event
%					\item {\bf $E_{1}$:}  Reaction of the environment to $S_{1}$
%					\item {\bf $S_{2}$:}  Response of the system to $E_{1}$
%					\item {\bf $E_{2}$:}  Reaction of the environment to $S_{2}$
%					\item[] $\cdots$
%					\item {\bf $S_{m}$:}  Response of the system to $E_{(m-1)}$
%					\item {\bf $E_{m}$:}  Reaction of the environment to $E_{(m-1)}$
%					\item {\bf $S_{(m+1)}$:} Final response of the system concluding its function regarding the Business Event
%				\end{itemize}
%			}%
%		}	
%		%\end{enumerate}
%		\item Business Event name
%		\begin{enumerate}[{VP1}.1]
%			\item Viewpoint name \newline
%			\noindent\fbox{%
%				\parbox{0.5\textwidth}{%
%					\begin{itemize}
%						\item {\bf $S_{1}$:} Initial response of the system to the Business Event
%						\item {\bf $E_{1}$:}  Reaction of the environment to $S_{1}$
%						\item {\bf $S_{2}$:}  Response of the system to $E_{1}$
%						\item {\bf $E_{2}$:}  Reaction of the environment to $S_{2}$
%						\item[] $\cdots$
%						\item {\bf $S_{n'}$:}  Response of the system to $E_{(n'-1)}$
%						\item {\bf $E_{n'}$:}  Reaction of the environment to $E_{(n'-1)}$
%						\item {\bf $S_{(n'+1)}$:} Final response of the system concluding its function regarding the Business Event
%					\end{itemize}
%				}%
%			}
%			\item Viewpoint name\newline
%			\noindent\fbox{%
%				\parbox{0.5\textwidth}{%
%					\begin{itemize}
%						\item {\bf $S_{1}$:} Initial response of the system to the Business Event
%						\item {\bf $E_{1}$:}  Reaction of the environment to $S_{1}$
%						\item {\bf $S_{2}$:}  Response of the system to $E_{1}$
%						\item {\bf $E_{2}$:}  Reaction of the environment to $S_{2}$
%						\item[] $\cdots$
%						\item {\bf $S_{k'}$:}  Response of the system to $E_{(k'-1)}$
%						\item {\bf $E_{k'}$:}  Reaction of the environment to $E_{(k'-1)}$
%						\item {\bf $S_{(k'+1)}$:} Final response of the system concluding its function regarding the Business Event
%					\end{itemize}
%				}%
%			}
%			\item \dots
%			\item \dots
%			\item \dots
%			\item[\dots]
%		\end{enumerate}	
%		\item[] {\bf Global Scenario of {\it Business Event Name}:} It is the scenario corresponding to the integration of all the above scenarios from the different Viewpoints of the Business Event BE2.\newline
%		\noindent\fbox{%
%			\parbox{0.5\textwidth}{%
%				\begin{itemize}
%					\item {\bf $S_{1}$:} Initial response of the system to the Business Event
%					\item {\bf $E_{1}$:}  Reaction of the environment to $S_{1}$
%					\item {\bf $S_{2}$:}  Response of the system to $E_{1}$
%					\item {\bf $E_{2}$:}  Reaction of the environment to $S_{2}$
%					\item[] $\cdots$
%					\item {\bf $S_{m'}$:}  Response of the system to $E_{(m'-1)}$
%					\item {\bf $E_{m'}$:}  Reaction of the environment to $E_{(m'-1)}$
%					\item {\bf $S_{(m'+1)}$:} Final response of the system concluding its function regarding the Business Event
%				\end{itemize}
%			}%
%		}		
%	\end{enumerate}

%End Section

\section{Non-Functional Requirements}
\label{sec:non-functional_requirements}


\begin{itemize}
	\item For each non-functional requirement, provide a justification/rationale for it.\\
	{\bf Example:} \\
	SC1. \emph{The device should not explode in a customer’s pocket.}\\
	{\bf Rationale:} Other companies have had issues with the batteries they used in their phones randomly exploding [insert citation]. This causes a safety issue, as the phone is often carried in a person's hand or pocket.	
	\item If you need to make a guess because you couldn't really talk to stakeholders, you can say "We imagined stakeholders would want...because..."
	\item Each requirement should have a unique label/number for it.
	\item In the list below, if a particular section doesn't apply, just write N/A so we know you considered it.
\end{itemize}

% Begin Section
\subsection{Look and Feel Requirements}
\label{sub:look_and_feel_requirements}
% Begin SubSection

\subsubsection{Appearance Requirements}
\label{ssub:appearance_requirements}
% Begin SubSubSection
\begin{enumerate}[{LF-A}1. ]
	\item 
\end{enumerate}
% End SubSubSection

\subsubsection{Style Requirements}
\label{ssub:style_requirements}
% Begin SubSubSection
\begin{enumerate}[{LF-S}1. ]
	\item 
\end{enumerate}
% End SubSubSection

% End SubSection

\subsection{Usability and Humanity Requirements}
\label{sub:usability_and_humanity_requirements}
% Begin SubSection

\subsubsection{Ease of Use Requirements}
\label{ssub:ease_of_use_requirements}
% Begin SubSubSection
\begin{enumerate}[{UH-EOU}1. ]
	\item 
\end{enumerate}
% End SubSubSection

\subsubsection{Personalization and Internationalization Requirements}
\label{ssub:personalization_and_internationalization_requirements}
% Begin SubSubSection
\begin{enumerate}[{UH-PI}1. ]
	\item 
\end{enumerate}
% End SubSubSection

\subsubsection{Learning Requirements}
\label{ssub:learning_requirements}
% Begin SubSubSection
\begin{enumerate}[{UH-L}1. ]
	\item 
\end{enumerate}
% End SubSubSection

\subsubsection{Understandability and Politeness Requirements}
\label{ssub:understandability_and_politeness_requirements}
% Begin SubSubSection
\begin{enumerate}[{UH-UP}1. ]
	\item 
\end{enumerate}
% End SubSubSection

\subsubsection{Accessibility Requirements}
\label{ssub:accessibility_requirements}
% Begin SubSubSection
\begin{enumerate}[{UH-A}1. ]
	\item 
\end{enumerate}
% End SubSubSection

% End SubSection

\subsection{Performance Requirements}
\label{sub:performance_requirements}
% Begin SubSection

\subsubsection{Speed and Latency Requirements}
\label{ssub:speed_and_latency_requirements}
% Begin SubSubSection
\begin{enumerate}[{PR-SL}1. ]
	\item 
\end{enumerate}
% End SubSubSection

\subsubsection{Safety-Critical Requirements}
\label{ssub:safety_critical_requirements}
% Begin SubSubSection
\begin{enumerate}[{PR-SC}1. ]
	\item 
\end{enumerate}
% End SubSubSection

\subsubsection{Precision or Accuracy Requirements}
\label{ssub:precision_or_accuracy_requirements}
% Begin SubSubSection
\begin{enumerate}[{PR-PA}1. ]
	\item 
\end{enumerate}
% End SubSubSection

\subsubsection{Reliability and Availability Requirements}
\label{ssub:reliability_and_availability_requirements}
% Begin SubSubSection
\begin{enumerate}[{PR-RA}1. ]
	\item 
\end{enumerate}
% End SubSubSection

\subsubsection{Robustness or Fault-Tolerance Requirements}
\label{ssub:robustness_or_fault_tolerance_requirements}
% Begin SubSubSection
\begin{enumerate}[{PR-RFT}1. ]
	\item 
\end{enumerate}
% End SubSubSection

\subsubsection{Capacity Requirements}
\label{ssub:capacity_requirements}
% Begin SubSubSection
\begin{enumerate}[{PR-C}1. ]
	\item 
\end{enumerate}
% End SubSubSection

\subsubsection{Scalability or Extensibility Requirements}
\label{ssub:scalability_or_extensibility_requirements}
% Begin SubSubSection
\begin{enumerate}[{PR-SE}1. ]
	\item 
\end{enumerate}
% End SubSubSection

\subsubsection{Longevity Requirements}
\label{ssub:longevity_requirements}
% Begin SubSubSection
\begin{enumerate}[{PR-L}1. ]
	\item 
\end{enumerate}
% End SubSubSection

% End SubSection

\subsection{Operational and Environmental Requirements}
\label{sub:operational_and_environmental_requirements}
% Begin SubSection

\subsubsection{Expected Physical Environment}
\label{ssub:expected_physical_environment}
% Begin SubSubSection
\begin{enumerate}[{OE-EPE}1. ]
	\item 
\end{enumerate}
% End SubSubSection

\subsubsection{Requirements for Interfacing with Adjacent Systems}
\label{ssub:requirements_for_interfacing_with_adjacent_systems}
% Begin SubSubSection
\begin{enumerate}[{OE-IA}1. ]
	\item 
\end{enumerate}
% End SubSubSection

\subsubsection{Productization Requirements}
\label{ssub:productization_requirements}
% Begin SubSubSection
\begin{enumerate}[{OE-P}1. ]
	\item 
\end{enumerate}
% End SubSubSection

\subsubsection{Release Requirements}
\label{ssub:release_requirements}
% Begin SubSubSection
\begin{enumerate}[{OE-R}1. ]
	\item 
\end{enumerate}
% End SubSubSection

% End SubSection

\subsection{Maintainability and Support Requirements}
\label{sub:maintainability_and_support_requirements}
% Begin SubSection

\subsubsection{Maintenance Requirements}
\label{ssub:maintenance_requirements}
% Begin SubSubSection
\begin{enumerate}[{MS-M}1. ]
	\item 
\end{enumerate}
% End SubSubSection

\subsubsection{Supportability Requirements}
\label{ssub:supportability_requirements}
% Begin SubSubSection
\begin{enumerate}[{MS-S}1. ]
	\item 
\end{enumerate}
% End SubSubSection

\subsubsection{Adaptability Requirements}
\label{ssub:adaptability_requirements}
% Begin SubSubSection
\begin{enumerate}[{MS-A}1. ]
	\item 
\end{enumerate}
% End SubSubSection

% End SubSection

\subsection{Security Requirements}
\label{sub:security_requirements}
% Begin SubSection

\subsubsection{Access Requirements}
\label{ssub:access_requirements}
% Begin SubSubSection
\begin{enumerate}[{SR-AC}1. ]
	\item The system must require all users of the dashboard to authenticate using an email/username and password, being granted access to the system. \\
    \textbf{Rationale}: This ensures that only people who are authorized can access the system, which prevents the risk of authorized individuals to alter environmental data [https://www.cyber.gc.ca/en/guidance/user-authentication-guidance-information-technology-systems-itsp30031-v3].

    \item The system must support multi-factor authentication (MFA) for administrative accounts. \\
    \textbf{Rationale}: MFA helps by eliminating the possibility of compromising the credentials of an administrators account. This is crucial, since the administrator is a high-privilege user who can define alert rules, and view sensitive system information [https://www.cyber.gc.ca/en/guidance/secure-your-accounts-and-devices-multi-factor-authentication-itsap30030].

    \item The system must enforce Role-Base Access Control (RBAC) to distinguish between the different user groups, such as administrator, city operator, and read-only public users. \\
    \textbf{Rationale}: Enforcing an RBAC policy ensures that each user group can only perform actions that are appropriate to their respective role. This helps to reduce accidental misuse of the systems functions. 

    \item The system must lock out a user account after five consecutive failed login attempts. \\
    \textbf{Rationale}: Protects against repeated attempts of trying to login to an account that does not belong to the respective individual [https://www.cyber.gc.ca/en/guidance/user-authentication-guidance-information-technology-systems-itsp30031-v3].

    \item All IoT devices must be authenticated with a unique key before sending telemetry data to the system. \\
    \textbf{Rationale}: Prevents harmful telemetry to be injection into the system, by limiting entering data to registered and authorized sensors [https://www.cyber.gc.ca/en/guidance/internet-things-iot-security-itsap00012].
\end{enumerate}
% End SubSubSection

\subsubsection{Integrity Requirements}
\label{ssub:integrity_requirements}
% Begin SubSubSection
\begin{enumerate}[{SR-INT}1. ]
    \item Any data transmitted between IoT devices, and the system must be encrypted, and include integrity protection. \\
    \textbf{Rationale}: Ensures that telemetry is protected, and cannot be changed during transmission. As a result, this helps maintain confidentiality and integrity [https://www.cyber.gc.ca/en/guidance/using-encryption-keep-your-sensitive-data-secure-itsap40016].
    
	\item All incoming sensor telemetry must be validated against a defined schema and acceptable value ranges before being stored. \\
    \textbf{Rationale}: Ensures that only correct and meaningful data is allowed to enter the system. This restricts any corrupted/malicious data from affecting any environmental analytics and alerts. 

    \item The system must maintain an audit log to keep track of any data modification, such as alert rule updates, or alert status changes. \\
    \textbf{Rationale}: An audit log is essential to provide traceability to identify any potential modifications made accidentally, or by unauthorized personnel [https://www.researchgate.net/publication/352745109\_Audit\_Logs\_Management\_and\_Security\_-\_A\_Survey].

    \item The system should use cryptographic hashing for any alert generated, to ensure that the alert data remains secure, and unaltered after generation. \\
    \textbf{Rationale}: Enforces data integrity, and prevents city operators from receiving false notifications [https://www.cyber.gc.ca/en/guidance/guidance-securely-configuring-network-protocols-itsp40062].
\end{enumerate}
% End SubSubSection

\subsubsection{Privacy Requirements}
\label{ssub:privacy_requirements}
% Begin SubSubSection
\begin{enumerate}[{SR-P}1. ]
	\item The system will not collect or store any personally identifiable information from public users. \\
    \textbf{Rationale}: Protects individual privacy, and ensures compliance with PIPEDA regulations [https://www.priv.gc.ca/en/privacy-topics/privacy-laws-in-canada/the-personal-information-protection-and-electronic-documents-act-pipeda/p\_principle/principles/p\_collection/]

    \item The system shall only collect environmental data at the level of general city zones. \\
    \textbf{Rationale}: Ensures citizen privacy by preventing sensor data from tracking individual residents

    \item Public APIs should only provide non-sensitive data to public citizens and third-party developers. \\
    \textbf{Rationale}: Prevents misuse of the API, and protects privacy.
    
\end{enumerate}
% End SubSubSection

\subsubsection{Audit Requirements}
\label{ssub:audit_requirements}
% Begin SubSubSection
\begin{enumerate}[{SR-AU}1. ]
	\item The system shall log all user login attempts, including the timestamp, their distinct identification number, and the location of the request. \\
    \textbf{Rationale}: Ensures accountability for access attempts [https://www.researchgate.net/publication/352745109\_Audit\_Logs\_Management\_and\_Security\_-\_A\_Survey].

    \item The system shall log all access to sensitive data, and data modification such as updating, adding or deleting rule alerts, or registering new sensors or accounts. \\
    \textbf{Rationale}: Ensures sensitive data is properly handled, and traceability of critical changes.[https://www.secoda.co/glossary/audit-traceability-audit-logs].

    \item All audit logs should be retained for at least 1 years. \\
    \textbf{Rationale}: Ensures there is enough time look at incidents, or check compliance [https://auditboard.com/blog/security-log-retention-best-practices-guide].

    \item All audit logs should be protected with access controls to ensure that only authorized security can view or extract information from them. \\
    \textbf{Rationale}: Prevents critical information from being shared without authorization.[https://auditboard.com/blog/security-log-retention-best-practices-guide].
    
\end{enumerate}
% End SubSubSection

\subsubsection{Immunity Requirements}
\label{ssub:immunity_requirements}
% Begin SubSubSection
\begin{enumerate}[{SR-IM}1. ]
	\item The system shall keep receiving and storing sensor data even when the real-time processing system is down. \\
    \textbf{Rationale}: Makes sure no sensor data is lost if part of the system stops.    [https://www.cyber.gc.ca/en/cyber-security-readiness/cyber-security-readiness-goals-securing-our-most-critical-systems].

    \item The system shall immediately try again when sending sensor data fails.\\
    \textbf{Rationale}: Ensures that sensor data eventually reaches the system, even in the case of network problems.    [https://www.cyber.gc.ca/en/cyber-security-readiness/cyber-security-readiness-goals-securing-our-most-critical-systems].


\end{enumerate}
% End SubSubSection

% End SubSection

\subsection{Cultural and Political Requirements}
\label{sub:cultural_and_political_requirements}
% Begin SubSection

\subsubsection{Cultural Requirements}
\label{ssub:cultural_requirements}
% Begin SubSubSection
\begin{enumerate}[{CP-C}1. ]
	\item The system shall not allow users to create an account with an inappropriate or discriminatory name. \\
    \textbf{Rationale}:

    \item The system shall support display of all dashboards of alerts in both French and English. \\
    \textbf{Rationale}:
    
    \item The system shall not use any symbols, colours, and icons in the dashboards that can be considered offensive based one ones religion, ethnicity, disability or sexual orientation.  \\
    \textbf{Rationale}:

    \item The system shall not use any symbols, colours, and icons in the dashboards that can be considered offensive based one ones religion, ethnicity, disability or sexual orientation.  \\
    \textbf{Rationale}:
\end{enumerate}
% End SubSubSection

\subsubsection{Political Requirements}
\label{ssub:political_requirements}
% Begin SubSubSection
\begin{enumerate}[{CP-P}1. ]
	\item The system shall comply with all federal, provincial, and municipal regulations regarding environmental monitoring and data sharing [https://www.canada.ca/en/environment-climate-change/services/canadian-environmental-protection-act-registry/monitoring-reporting-research/monitoring.html]. \\
    \textbf{Rationale}: Ensures legal compliance and helps to avoid political conflicts.
    	
    \item The system shall be transparent in how it collects and stores data by giving government authorities access to such data and system reports [https://www.tbs-sct.canada.ca/pol/doc-eng.aspx?id=28108]. \\
    \textbf{Rationale}: Enables accountability to political authorities. 
        	
    \item The system shall provide ways to share environmental data to other authorized groups. \\
    \textbf{Rationale}: Supports collaboration between government organizations.
\end{enumerate}
% End SubSubSection

\subsection{Legal Requirements}
\label{sub:legal_requirements}
% Begin SubSection

\subsubsection{Compliance Requirements}
\label{ssub:compliance_requirements}
% Begin SubSubSection
\begin{enumerate}[{LR-COMP}1. ]
	\item 
\end{enumerate}
% End SubSubSection

\subsubsection{Standards Requirements}
\label{ssub:standards_requirements}
% Begin SubSubSection
\begin{enumerate}[{LR-STD}1. ]
	\item 
\end{enumerate}
% End SubSubSection

% End SubSection

% End Section

\appendix
\section{Division of Labour}
\label{sec:division_of_labour}
% Begin Section
Include a Division of Labour sheet which indicates the contributions of each team member. This sheet must be signed by all team members.
% End Section

%\newpage
%\section*{IMPORTANT NOTES}
%\begin{itemize}
%	\item Be sure to include all sections of the template in your document regardless whether you have something to write for each or not
%	\begin{itemize}
%		\item If you do not have anything to write in a section, indicate this by the \emph{N/A}, \emph{void}, \emph{none}, etc.
%	\end{itemize}
%	\item Uniquely number each of your requirements for easy identification and cross-referencing
%	\item Highlight terms that are defined in Section~1.3 (\textbf{Definitions, Acronyms, and Abbreviations}) with \textbf{bold}, \emph{italic} or \underline{underline}
%	\item For Deliverable 1, please highlight, in some fashion, all (you may have more than one) creative and innovative features. Your creative and innovative features will generally be described in Section~2.2 (\textbf{Product Functions}), but it will depend on the type of creative or innovative features you are including.
%\end{itemize}


\end{document}
%------------------------------------------------------------------------------